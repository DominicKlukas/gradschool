\documentclass[11pt]{article}
\usepackage[margin=1in]{geometry}
\usepackage{amsmath,amssymb,mathtools}
\usepackage{hyperref}
\usepackage{enumitem}
\usepackage{graphicx}
\usepackage{parskip}
\setlength{\parindent}{0pt}
\setlength{\parskip}{6pt}

% Some basic packages
\usepackage[utf8]{inputenc}
\usepackage[T1]{fontenc}
\usepackage{textcomp}
\usepackage{url}
\usepackage{graphicx}
\usepackage{float}
\usepackage{booktabs}
\usepackage{enumitem}
\usepackage{bbm}

\pdfminorversion=7

% Don't indent paragraphs, leave some space between them
\usepackage{parskip}

% Hide page number when page is empty
\usepackage{emptypage}
\usepackage{subcaption}
\usepackage{multicol}
\usepackage{xcolor}

% Other font I sometimes use.
% \usepackage{cmbright}

% Math stuff
\usepackage{amsmath, amsfonts, mathtools, amsthm, amssymb}
% Fancy script capitals
\usepackage{mathrsfs}
\usepackage{cancel}
% Bold math
\usepackage{bm}
% Some shortcuts
\newcommand\N{\ensuremath{\mathbb{N}}}
\newcommand\R{\ensuremath{\mathbb{R}}}
\newcommand\Z{\ensuremath{\mathbb{Z}}}
\renewcommand\O{\ensuremath{\emptyset}}
\newcommand\Q{\ensuremath{\mathbb{Q}}}
\newcommand\C{\ensuremath{\mathbb{C}}}

\DeclareMathOperator{\var}{Var}
\DeclareMathOperator{\ani}{Ann}
\DeclareMathOperator{\mnn}{Min}
\DeclareMathOperator{\spn}{span}
\DeclareMathOperator{\ch}{ch}
\DeclareMathOperator{\aut}{Aut}
\DeclareMathOperator{\ide}{Id}
\DeclareMathOperator{\iso}{Iso}
\DeclareMathOperator{\sinc}{sinc}


% Easily typeset systems of equations (French package)
\usepackage{systeme}

% Put x \to \infty below \lim
\let\svlim\lim\def\lim{\svlim\limits}

%Make implies and impliedby shorter
\let\implies\Rightarrow
\let\impliedby\Leftarrow
\let\iff\Leftrightarrow
\let\epsilon\varepsilon

% Add \contra symbol to denote contradiction
\usepackage{stmaryrd} % for \lightning
\newcommand\contra{\scalebox{1.5}{$\lightning$}}

% \let\phi\varphi

% Command for short corrections
% Usage: 1+1=\correct{3}{2}

\definecolor{correct}{HTML}{009900}
\newcommand\correct[2]{\ensuremath{\:}{\color{red}{#1}}\ensuremath{\to }{\color{correct}{#2}}\ensuremath{\:}}
\newcommand\green[1]{{\color{correct}{#1}}}

% horizontal rule
\newcommand\hr{
    \noindent\rule[0.5ex]{\linewidth}{0.5pt}
}

% hide parts
\newcommand\hide[1]{}

% si unitx
\usepackage{siunitx}
\sisetup{locale = US}

% Environments
\makeatother
% For box around Definition, Theorem, \ldots
\usepackage{mdframed}
\mdfsetup{skipabove=1em,skipbelow=0em}
\theoremstyle{definition}
\newmdtheoremenv[nobreak=true]{definition}{Definition}
\newmdtheoremenv[nobreak=true]{property}{Property}
\newmdtheoremenv[nobreak=true]{consequence}{Consequence}
\newmdtheoremenv[nobreak=true]{lemma}{Lemma}
\newmdtheoremenv[nobreak=true]{proposition}{Proposition}
\newmdtheoremenv[nobreak=true]{theorem}{Theorem}
\newmdtheoremenv[nobreak=true]{law}{Law}
\newmdtheoremenv[nobreak=true]{corollary}{Corollary}
\newmdtheoremenv{conclusion}{Conclusion}
\newmdtheoremenv{bonus}{Bonus}
\newmdtheoremenv{conjecture}{Conjecture}
\newtheorem*{recall}{Recall}
\newtheorem*{intermezzo}{Intermezzo}
\newtheorem*{notation}{Notation}
\newtheorem*{observation}{Observation}
\newtheorem*{exercise}{Exercise}
\newtheorem*{comment}{Comment}
\newtheorem*{practice}{practice}
\newtheorem*{problem}{Problem}
\newtheorem*{terminology}{Terminology}
\newtheorem*{application}{Application}
\newtheorem*{eg}{Example}
\newtheorem*{question}{Question}
\newtheorem*{previouslyseen}{As previously seen}
\newtheorem*{remark}{Remark}
\newtheorem*{observe}{Observe}
\newtheorem*{intuition}{Intuition}
\newtheorem*{solution}{Solution}

% End example and intermezzo environments with a small diamond (just like proof
% environments end with a small square)
\usepackage{etoolbox}
\AtEndEnvironment{vb}{\null\hfill$\diamond$}%
\AtEndEnvironment{intermezzo}{\null\hfill$\diamond$}%
% \AtEndEnvironment{opmerking}{\null\hfill$\diamond$}%

% Fix some spacing
% http://tex.stackexchange.com/questions/22119/how-can-i-change-the-spacing-before-theorems-with-amsthm
\makeatletter
\def\thm@space@setup{%
  \thm@preskip=\parskip \thm@postskip=0pt
}


% Exercise 
% Usage:
% \oefening{5}
% \suboefening{1}
% \suboefening{2}
% \suboefening{3}
% gives
% Oefening 5
%   Oefening 5.1
%   Oefening 5.2
%   Oefening 5.3
%\newcommand{\exc}[1]{%
    %\def\@excercise{#1}%
    %\subsection*{Excercise #1}
%}

%\newcommand{\sub-excercise}[1]{%
    %\subsubsection*{Excercise \@excercise.#1}
%}


% \lecture starts a new lecture (les in dutch)
%
% Usage:
% \lecture{1}{di 12 feb 2019 16:00}{Inleiding}
%
% This adds a section heading with the number / title of the lecture and a
% margin paragraph with the date.

% I use \dateparts here to hide the year (2019). This way, I can easily parse
% the date of each lecture unambiguously while still having a human-friendly
% short format printed to the pdf.

%\usepackage{xifthen}
%\def\testdateparts#1{\dateparts#1\relax}
%\def\dateparts#1 #2 #3 #4 #5\relax{
    %\marginpar{\small\textsf{\mbox{#1 #2 #3}}}
%}

%\def\@lecture{}%
\newcommand{\note}[3]{
    \ifthenelse{\isempty{#3}}{%
        \def\@note{#1}%
    }{%
        \def\@note{#1: #3}%
    }%
    \subsection*{\@note}
    \marginpar{\small\textsf{\mbox{#2}}}
}



% These are the fancy headers

\makeatother


% Todonotes and inline notes in fancy boxes
\usepackage{tcolorbox}

% Make boxes breakable
\tcbuselibrary{breakable}

% Verbetering is correction in Dutch
% Usage: 
% \begin{verbetering}
%     Lorem ipsum dolor sit amet, consetetur sadipscing elitr, sed diam nonumy eirmod
%     tempor invidunt ut labore et dolore magna aliquyam erat, sed diam voluptua. At
%     vero eos et accusam et justo duo dolores et ea rebum. Stet clita kasd gubergren,
%     no sea takimata sanctus est Lorem ipsum dolor sit amet.
% \end{verbetering}
\newenvironment{correction}{\begin{tcolorbox}[
    arc=0mm,
    colback=white,
    colframe=green!60!black,
    title=Remark,
    fonttitle=\sffamily,
    breakable
]}{\end{tcolorbox}}



% Figure support as explained in my blog post.
\usepackage{import}
\usepackage{xifthen}
\usepackage{pdfpages}
\usepackage{transparent}
\newcommand{\incfig}[1]{%
    \def\svgwidth{\columnwidth}
    \import{./figures/}{#1.pdf_tex}
}

% Fix some stuff
% %http://tex.stackexchange.com/questions/76273/multiple-pdfs-with-page-group-included-in-a-single-page-warning
\pdfsuppresswarningpagegroup=1

% My name
\author{Dominic Klukas}


\newcommand{\coursename}{EECE 571N -- Sequential Decision-Making (EECE 571N)}
\newcommand{\assignmenttitle}{HW 1: MDPs, Policy Iteration, and Value Iteration}
\newcommand{\instructorname}{Cyrus Neary}
\newcommand{\duedate}{2025-09-29 at 23:59 PT}

\begin{document}
\begin{center}
{\Large \textbf{\assignmenttitle}}\\
\coursename\\[4pt]
\textbf{Instructor: \instructorname}\\
\textbf{Due: \duedate}
\end{center}

\textbf{Name:} Dominic Klukas
\hfill
\textbf{Student Number:} 64348378

\section*{Instructions}
Submit a single PDF to Canvas. Please include your name and student number at the top of that PDF. For all questions, show your work and clearly justify your steps and thinking. Please feel free to include any code as an attachment at the end of the PDF. State any assumptions. Unless otherwise specified, you may collaborate conceptually but must write up your own solutions independently.

\section*{Grading}
Points for each part are indicated. The total number of achievable points is 100. Partial credit is available for incorrect answers with clear reasoning.

% --- Begin your solutions below ---
% (Leave blank or add your own sections as needed.)
\section*{Problem 1}
$V^{\pi}(s)$ represents the expected reward that a policy $\pi$ will earn in a Markov Decision Process if it starts in state $s$. 
$V^{*}(s)$ represents the expected reward that an optimal policy will earn in a Markov Decision Process if it starts in state $s$.
Bellman's equation for $V^{\pi}(s)$ is given by
\[
V^{\pi}(s) = \sum_{a \in A} \pi (a | s) \left( R(s, a) + \gamma \cdot \sum_{s' \in S} T(s' | s, a) V^{\pi}(s')  \right)
.\] 
Bellman's optimality equation for $V^{*}(s)$ is given by
\[
V^{*}(s) = \max_{a \in A}\left( R(s, a) + \gamma \cdot \sum_{s' \in S} T(s' | s, a) V^{*}(s')  \right)
.\] 

\section*{Problem 2}
Now, suppose $V$ and $W$ are any two functions of $s$.
Let $\pi$ be any policy.
First, we compute:
\begin{align*}
    (T_{\pi} V - T_{\pi} W)(s) &=\sum_{a \in A} \pi(a|s)\left(R(s, a) + \gamma \sum_{s' \in S}T(s' | s, a) V(s')\right) \\
                          & \, - \sum_{a \in A} \pi(a|s)\left(R(s, a) + \gamma \sum_{s' \in S}T(s' | s, a) W(s')\right) \\
                          &= \sum_{a \in A} \pi(a|s)\left(\gamma \sum_{s' \in S} T(s' | s, a) (V(s') - W(s'))\right)
.\end{align*}

From the definition of the $\|\cdot \|_{\infty}$ norm as $\sup_{s \in S} |f(s)|$, and applying the traingle inequality, we have
\begin{align*}
    \|T_{\pi} V - T_{\pi} W \|_{\infty} &= \sup_{s \in S} |T_{\pi} V(s) - T_{\pi} W(s)| \\
    &\le \sup_{s \in S}  \sum_{a \in A} \gamma \cdot  \pi(a | s) \sum_{s' \in S} T(s'|s, a) \left| V(s') - W(s')  \right| \text{ by triangle inequality}  \\
    &\le \gamma \cdot \sup_{s \in S} \sum_{a \in A} \pi(a | s) \sum_{s' \in S} T(s' | s, a) \sup_{s''\in S} |V(s'') - W(s'')|
.\end{align*}
But then, $\sup_{s'' \in S} |V(s'') - W(s'')| = \|V - W\|_{\infty}$ which is just a constant, and $\sum_{s'\in S} T(s'|s, a) = 1$ since it is a probability distribution. Likewise, $\sum_{a \in A} \pi(a | s) = 1$. Putting these together:

\begin{align*}
    \|T_{\pi} V - T_{\pi} W\|_{\infty} &\le  \gamma \cdot  \sup_{s \in S} \sum_{a \in A} \pi(a | s) \sum_{s' \in S} T(s' | s, a)\|V - W\|_{\infty}\\
    \|T_{\pi} V - T_{\pi} W\|_{\infty} &\le  \gamma \cdot \|V - W\|_{\infty}\cdot  \sup_{s \in S} \sum_{a \in A} \pi(a | s)\cdot 1\\
    \|T_{\pi} V - T_{\pi} W\|_{\infty} &\le  \gamma \cdot \|V - W\|_{\infty} \cdot  \sup_{s \in S} (1) = \gamma \cdot  \|V - W\|_{\infty}
,\end{align*}
as desired.
\section*{Problem 3}
The statement of the Banach fixed point theorem, essentially verbatim from Wikipedia with small modifications to fit our notations, is as follows:

Let $(X, d)$ be a non-empty complete metric space with a $\gamma$-contraction mapping $T: X \to X$.
Then $T$ admits a unique fixed-point $x^{*}$ in $X$ (i.e. $T(x^*) = x^*$).
Furthermore, $x^*$ can be found as follows: start with an arbitrary element $x_0 \in X$ and define a sequence $(x_n)_{n \in \N}$ by $x_n = T(x_{n-1})$ for $n \ge  1$.
Then, $\lim_{n \to \infty} x_{n} = x^{*} $.
Furthermore, the speed of convergence is bounded in the sense that $d(x^{*}, x_{n}) \le  \frac{\gamma^{n}}{1 - \gamma} d(x_1, x_0)$.
\section*{Problem 4}
In order to apply the Banach fixed point theorem on an operator $T$, we have to make sure $(X, d)$ which it is acting on is a metric space, and with that, a complete metric space.
In our case, $X$ is the set of functions $V : S \to \R$, and $d$ is $\|\cdot \|_{\infty}$.
Since $S$ is finite, $V: S \to \R$ is clealry isomorphic to $\mathbb{R}^{|S|}$.
But then:
\begin{proof}
    Let $\{x_n\}_{n \in \N} \in \R^{d} $ be a Cauchy sequence with respect to the $\|\cdot \|_{\infty}$ norm.
    We prove that this sequence is also a pointwise Cauchy sequence.
    Fix some index $i$.
    Let $\epsilon > 0$.
    Then, there exists some $N$ such that for all $n, m > N$, we have $\|x_n - x_m\|_{\infty} < \epsilon$.
    By the definition of the $\infty$-norm, it follows that for each index $1 \le  j \le  N$, $|(x_n)_j - (x_m)_j| < \epsilon$.
    In particular, $\{(x_n)_i\}_{n \in \N}$ is a Cauchy sequence.
We proved in Math 320 that Cauchy sequences converge in $\R$: the proof works by showing that first, the sequences are bounded. Next, every bounded sequence has a convergent subsequence (this result uses compactness), and finally, that the whole sequence converges to the same limit as this subsequence.
This pointwise convergence, $\{(x_n)_i\}_{n \in \N} \to x_i \in \R $, implies $x_n = ((x_n)_{1}, \ldots, (x_n)_d) \to (x_1, \ldots, x_d) \in \R^{d}$, as desired.
\end{proof}

Furthermore, $V(s) = 0$ is a valid function in our space, so $X$ is non-empty.

Finally $T_{\pi}$ is indeed a $\gamma$-contraction mapping, as we proved in Question 2. Therefore, the requirements of the Banach fixed point theorem are met and we can use the theorem: a unique fixed point exists. That is, there exists a function $V : S \to \R$ such that
\[
T_{\pi}V(s) = \sum_{a \in A} \pi(a | s) \left( R(s, a) + \gamma \sum_{s' \in S} T(s' | s, a) V(s') \right) 
\] 
($V$ is a fixed point of $T_{\pi}$), and $V$ is the limit of any sequence given by $V_{n+1} = T_{\pi} V_n$ initialized by any bounded function $V_0 : S \to \R$.

However, the equation satisfied by this fixed point is precisely the equation satisfied by the value function $V^{\pi}$ for policy $\pi$, so we conclude that the value function is the function that the contraction mapping converges to, regardless of the initial function $V_0$ in the sequence.

\section*{Problem 5}
We define each component of the MDP in turn.
\begin{itemize}
    \item $S$

        We define the state by the Robot's location on the gridworld.  Every location is possible except for the impassable wall.
        We enumerate the states by their co-ordinates, where $n$ is the row from the top and $m$ is the column from the left.
        Thus,
        \[
        S = \{(n, m)\}_{\substack{1 \le  n \le  5\\ 1 \le  m \le  4}} \setminus \{(2, 3)\}
        .\] 

    \item $A$

        At each state, we have the set of actions $\{\text{left}, \text{right}, \text{up}, \text{down}\} $.
        To make these easier to work with, define $A = \{(-1, 0), (1, 0), (0, 1), (0, -1)\} $.
        Note: actions that result in moving into the wall are allowed, as specified in the question.
    \item $T$

        For $T$, we define the set $N(s) = \{s + (1, 0), s + (0, 1), s + (-1, 0), s + (0, -1)\} \cap S $.
        In other words, this is the set of valid "neighbors" in the set $S$ that the robot can move to from $s$.
        \begin{align*}
            T(s' | s, a) = \begin{cases}
                1 & \text{ if } s' = s \text{ and } s \in \{(5, 4), (5, 1)\} \\
                0 & \text{ if } s' \neq s  \text{ and } s \in \{(5, 4), (5, 1)\} \\
                (1-p) & \text{ if } s' = s + a \text{ and } s' \in N(s), s \not\in  \{(5, 4), (5, 1)\} \\
                (1-p) & \text{ if } s' = s \text{ and } s + a \not\in  N(s), s \not\in \{(5, 4), (5, 1)\} \\
                p / |N(s)| & \text{ if } s' \in N(s), s \not\in \{(5, 4), (5, 1)\} \\
                0 & \text{ else}
            \end{cases}
        .\end{align*}
        The first two lines describe the case where $s$ is in one of the sink states.
        The third line describes the case when $s$ is not in a sink state, and the robot wants to move to a valid neighbor.
        The fourth line describes the case when $s$ is not in a sink state and the robot is trying to move into a wall.
        The fifth line describes the case when $s$ is not in a sink state and the robot is slipping.
        The last case is a catch all case.
    \item $\gamma$

        Since this is a finite horizon MDP we set $\gamma = 1$.
    \item $R$

        We can express the reward function for the problem in the form $R(s, a)$ with the following expression:
        \[
        R(s, a) = \sum_{s' \in S} T(s'|s, a) R(s, a, s')
        .\] 
        Then, we define $R(s, a, s')$:
        \begin{align*}
            R(s, a, s') &= \begin{cases}
                1 &\text{ if } s' = (1, 5)\\
                -1 &\text{ if } s' = (4, 5) \\
                0 &\text{ if } s \in \{(1, 4), (4, 4)\}\\
                -0.1 &\text{ else}
            \end{cases}
        .\end{align*}

    \item $\mu$

        This is the initial state function. We put the robot at its starting position, so that we have: $\mu(s) = \begin{cases}
            1 & \text{ if } s = (1, 4)\\
            0 & \text{ else}
        \end{cases}$

\end{itemize}
\section*{Problem 6}

Credit: I wrote all of the code myself, except for the code to generate the figures. See page 5.
% Your written solution goes here...

\begin{figure}[h]
  \centering
  \includegraphics[width=0.8\textwidth]{Problem6.png}
  \caption{Heatmap for the computed value function $V^\pi$ from the Policy Evaluation Iteration algorithm. $\pi$ is the policy that assigns a uniform probability distribution over the set of actions at each state. Since the time horizon is infinite, we chose $\gamma = 0.9$.}
  \label{fig:problem6}
\end{figure}

\clearpage
\section*{Problem 7}

% Your written solution goes here...

\begin{figure}[h]
  \centering
  \includegraphics[width=0.8\textwidth]{Problem7.png}
  \caption{Heatmap for the the converged optimal value function $V^*$ from the policy iteration algorithm, where the initial policy is $\pi$ from Problem 6 and the initial $V(s) = 0$. Since the time horizon is technically infinite, we chose $\gamma = 0.9$.}
  \label{fig:problem7}
\end{figure}

% \section*{Problem 6}
% \section*{Problem 7}

\end{document}
