\documentclass[11pt]{article}
\usepackage{graphicx} % Required for inserting images

%% Sets page size and margins
\usepackage[letterpaper, headheight=14pt, headsep=14pt, top=1.0in, margin=1.0in, bottom=1.0in]{geometry}

% \usepackage{times}

\usepackage{enumitem}

\usepackage{array}
\usepackage{fancyhdr}
\usepackage[normalem]{ulem}

% \setlength{\headsep}{10pt}

\usepackage{color}
\usepackage{wrapfig}
\usepackage{graphicx} % Required for inserting images
\usepackage{amssymb}

\usepackage{parskip}
\usepackage{titlesec}
\usepackage{lipsum}
\usepackage{adjustbox}

% Adjust the line spacing in the overall document.
\linespread{1.0}

% \setlength{\parskip}{0pt}      % No space between paragraphs
% \setlength{\parindent}{1.5em}    % Small indent at the start of each paragraph

% % Load caption package, and set skip (between figure and caption)
% \usepackage[skip=5pt]{caption}
% \setlength{\textfloatsep}{1mm}  % Adjust the value as needed, e.g., 5pt

% \setlength{\parskip}{4pt plus 1.0pt minus 2.0pt}

% Change the line spacing around paragraphs
\titlespacing{\paragraph}{%
  0pt}{%             left margin
  1.0ex}{%          space before (vertical)
  1em}%             space after (horizontal)

% Change the line spacing around section titles.
\titlespacing{\section}{
    0pt}{% Left margin
    10pt plus 4pt minus 2pt}{% Space before (vertical)
    6pt plus 2pt minus 2pt}% Space after (vertical)

% % Change the font size of section headings
% \titleformat{\section}{
%     \normalfont\fontsize{12}{15}\bfseries}{
%     \thesection}{
    % 1em}{}

%%%%%%%%%%%%%%% TIKZ/PLOTTING
\usepackage{pgfplots}
\usepgfplotslibrary{groupplots,dateplot}
\usetikzlibrary{automata,patterns,shapes.arrows}
\pgfplotsset{compat=newest}
\usepackage{tikz}
\usetikzlibrary{shapes, arrows.meta, positioning}
\usetikzlibrary{decorations.pathreplacing,calligraphy, shapes.callouts}

\usepgfplotslibrary{groupplots}
\usepgfplotslibrary{fillbetween}
\usetikzlibrary{arrows,decorations.pathmorphing,positioning,fit,trees,shapes,shadows,automata,calc} 
\usetikzlibrary{3d,patterns,arrows,arrows.meta,calc,shapes,shadows,decorations.pathmorphing,decorations.pathreplacing,automata,shapes.multipart,positioning,shapes.geometric,fit,circuits,trees,shapes.gates.logic.US,fit, matrix,arrows.meta, quotes}
% \tikzexternalize[prefix=figures/]
\usetikzlibrary{backgrounds,scopes}
\usepackage{neuralnetwork}
\usepackage{listofitems} % for \readlist to create arrays

%%%%%%%%%%%%%%%%%%%%%%%%%%%%%% CUSTOM LEGEND SETUP%%%%%%%%%%%%%%%%%%%%
\usepgfplotslibrary{external}
\pgfplotsset{compat=newest}
\newenvironment{customlegend}[1][]{%
	\begingroup
	% inits/clears the lists (which might be populated from previous
	% axes):
	\csname pgfplots@init@cleared@structures\endcsname
	\pgfplotsset{#1}%
}{%
	% draws the legend:
	\csname pgfplots@createlegend\endcsname
	\endgroup
}%

% makes \addlegendimage available (typically only available within an
% axis environment):
\def\addlegendimage{\csname pgfplots@addlegendimage\endcsname}
%%%%%%%%%%%%%%%%%%%%%%%%%%%%%%%%%%%%%%%%%%%%%%%%%%%%%%%%%%%%%%%%%%%%%%%
%%%%%%%%%%%%%%%%%%%%%%%% DEFINE TIKZ LAYERS %%%%%%%%%%%%%%%%%%%%%%%%%%%%
\pgfdeclarelayer{background}
\pgfdeclarelayer{background2}
\pgfdeclarelayer{background1}
\pgfdeclarelayer{foreground1}
\pgfdeclarelayer{foreground2}
\pgfsetlayers{background,background2,background1,main,foreground1,foreground2}

%%%%%%%%%%%% Bibliography setup
\usepackage[
    backend=bibtex,
    style=authoryear, 
    bibencoding=ascii,
    maxbibnames=5,
    sorting=nyt,
    giveninits=true,
    % useprefix=false,
]{biblatex}
\addbibresource{bibliography.bib}

% Customize name format to show only last names and first initials
% \DeclareNameFormat{author}{%
%   \usebibmacro{name:last}{#1}{#3}{#5}{#7}% Last name
%   \ifblank{#3}{}{\addspace\usebibmacro{name:first}{#1}{#3}{#5}{#7}}% First initial
% }

% % Map the `boldnumber` field for conditional bold formatting
% \DeclareSourcemap{
%   \maps[datatype=bibtex]{
%     \map{
%       \step[fieldsource=boldnumber, match=true]
%       \step[fieldset=usera, fieldvalue=true]
%     }
%   }
% }

% % Conditional formatting for bolding only the original number
% \DeclareFieldFormat{labelnumber}{%
%   \ifboolexpr{ test {\iffieldundef{usera}} }
%     {#1}% Regular number if usera is undefined
%     {\textbf{#1}}% Bold number if usera is defined
% }

\setlength\bibitemsep{0.00\baselineskip}
% You can define useful macros and latex commands here.

\newcommand{\mdp}{\mathcal{M}}
\newcommand{\state}{s}
\newcommand{\stateSpace}{S}
\newcommand{\action}{a}
\newcommand{\actionSet}{A}
\newcommand{\transitionFunction}{T}
\newcommand{\rewardFunction}{R}
\newcommand{\discount}{\gamma}

\begin{document}

\pagestyle{fancy}
\fancyhead{}
\fancyfoot{}
\fancyhead[L]{[Short Title] Project Proposal Template} % List title of the project proposal here.
\fancyhead[R]{Cyrus Neary} % List names of team members here.

\section{Motivation and Problem Statement}

Briefly describe the challenge you plan to address.
Elaborate on the potential impacts that solving this challenge could have on a specific industry, societal group, or specific research community.
While writing this section, think to yourselves ``If this project were to turn into a research paper, who would the target audience be?"

State the technical problem(s) that this project will solve in order to address the challenge described above.
There is no need to go into deep mathematical detail here, or to describe all the necessary background---only go into the detail that is necessary to have a precise problem statement that could be understood by someone who already has the necessary background knowledge. 

Include a figure to help set the stage if one is relevant.

\section{Background and Related Work}

Provide a brief survey of relevant prior work.
What are the pre-existing concepts, theory, and algorithms that are absolutely necessary to understanding your proposed approach?
What are some related approaches to solving technical problems similar to the ones that you plan to address? Try and include at least 3-5 related works. Of course, more than this is fine as well.

You can use the \verb|\parencite| command to include citations included at the end of sentences.
For example, here is a sentence about something interesting in reinforcement learning \parencite{sutton2018reinforcement}.
Alternatively, you can use the \verb|\textcite| command to include the names of the authors being cited as a subject in your sentence.
For example, \textcite{sutton2018reinforcement} wrote the definite textbook on the topic of reinforcement learning.

\section{Proposed Approach}

How will you accomplish the project's goals?
What will the core technical and conceptual contributions of this project be?
% For each aspect of the planned approach, give technical justifications for taking this approach?
Give technical justifications for each aspect of the planned approach.

Clearly state the planned contribution(s) of this project.
Be as specific as possible.
For each contribution or benefit of the proposed approach, describe the example(s), simulation(s) and/or experiment(s) that will showcase this idea clearly.

Include a figure to help describe the planned approach at a high level if one is relevant.


\section{Timeline and Division of Work}

Provide a detailed breakdown of the project into several major milestones.
What technical components will each milestone require?
What is the planned division of labour amongst the teammates for each of the planned milestones?
Give target deadlines for when each milestone be completed.

Give a brief justification that the planned experiments, simulations, or implemented examples will be feasible to complete within the scope of the semester, and with the planned project resources.

\printbibliography

\color{blue}
\section*{Grading Rubric}
The entire project proposal should be 2-3 pages in length, with unlimited extra space for references.
It is fine to go a little over the page limit, especially if you include figures that take up a large chunk of space.
However, please try to not go too far beyond the 3 page limit.
\begin{itemize}
    \item (20\%) \textbf{Motivation and Problem Statement} Clear consideration of a challenge to be overcome, and of what the impacts of overcoming that challenge would be. Clarity of problem statement.
    \item (20\%) \textbf{Background and Related Work} Evidence of having read relevant prior work.
    \item (30\%) \textbf{Proposed Approach} Does the proposed approach address the technical problems described above? Is the proposed approach clearly written and feasible?
    \item (20\%) \textbf{Timeline and Division of Work} Are the milestones clearly described? Is the proposed timeline realistic and specific? Is the work divided roughly evenly between the team members?
    \item (10\%) \textbf{References} Do the references follow a style that is self-consistent?
\end{itemize}

\section*{Helpful Prompts}
Here is a list of helpful questions that I like to keep in mind as I'm developing a research project.
There is no need to have concrete answers to each of these before writing the proposal or starting on the project itself.
However, I often find that the sooner I can answer each of the questions listed below, the more efficient and focused my work becomes.

\begin{enumerate}
    \item What problem does this project solve?
    \item Who is the target audience of this work? What sub-community will be interested in this work?
    \item What are the core technical and conceptual contributions?
    \item What are the technical justifications for taking this approach?
    \item What can we now do that wasn’t possible before? What are the benefits of this method? (For the course project, don't worry so much about novelty.)
    \item What are some previous methods and why couldn’t they do the same things? (Again, for the course project, don't worry so much about novelty with respect to prior work.)
    \item For each contribution / benefit of the method(s), describe the example(s), simulation(s) and/or experiment(s) that will showcase this benefit clearly.
    \item Why would somebody else use this work or build on it?
    \item What are the limitations of the work, and for each limitation, how will it be addressed?
    \item What is the single most important point that you want a reader to take away from a paper on this project?
\end{enumerate}
\color{black}


\end{document}
