%% Sets page size and margins
\usepackage[letterpaper, headheight=14pt, headsep=14pt, top=1.0in, margin=1.0in, bottom=1.0in]{geometry}

% \usepackage{times}

\usepackage{enumitem}

\usepackage{array}
\usepackage{fancyhdr}
\usepackage[normalem]{ulem}

% \setlength{\headsep}{10pt}

\usepackage{color}
\usepackage{wrapfig}
\usepackage{graphicx} % Required for inserting images
\usepackage{amssymb}

\usepackage{parskip}
\usepackage{titlesec}
\usepackage{lipsum}
\usepackage{adjustbox}

% Adjust the line spacing in the overall document.
\linespread{1.0}

% \setlength{\parskip}{0pt}      % No space between paragraphs
% \setlength{\parindent}{1.5em}    % Small indent at the start of each paragraph

% % Load caption package, and set skip (between figure and caption)
% \usepackage[skip=5pt]{caption}
% \setlength{\textfloatsep}{1mm}  % Adjust the value as needed, e.g., 5pt

% \setlength{\parskip}{4pt plus 1.0pt minus 2.0pt}

% Change the line spacing around paragraphs
\titlespacing{\paragraph}{%
  0pt}{%             left margin
  1.0ex}{%          space before (vertical)
  1em}%             space after (horizontal)

% Change the line spacing around section titles.
\titlespacing{\section}{
    0pt}{% Left margin
    10pt plus 4pt minus 2pt}{% Space before (vertical)
    6pt plus 2pt minus 2pt}% Space after (vertical)

% % Change the font size of section headings
% \titleformat{\section}{
%     \normalfont\fontsize{12}{15}\bfseries}{
%     \thesection}{
    % 1em}{}

%%%%%%%%%%%%%%% TIKZ/PLOTTING
\usepackage{pgfplots}
\usepgfplotslibrary{groupplots,dateplot}
\usetikzlibrary{automata,patterns,shapes.arrows}
\pgfplotsset{compat=newest}
\usepackage{tikz}
\usetikzlibrary{shapes, arrows.meta, positioning}
\usetikzlibrary{decorations.pathreplacing,calligraphy, shapes.callouts}

\usepgfplotslibrary{groupplots}
\usepgfplotslibrary{fillbetween}
\usetikzlibrary{arrows,decorations.pathmorphing,positioning,fit,trees,shapes,shadows,automata,calc} 
\usetikzlibrary{3d,patterns,arrows,arrows.meta,calc,shapes,shadows,decorations.pathmorphing,decorations.pathreplacing,automata,shapes.multipart,positioning,shapes.geometric,fit,circuits,trees,shapes.gates.logic.US,fit, matrix,arrows.meta, quotes}
% \tikzexternalize[prefix=figures/]
\usetikzlibrary{backgrounds,scopes}
\usepackage{neuralnetwork}
\usepackage{listofitems} % for \readlist to create arrays

%%%%%%%%%%%%%%%%%%%%%%%%%%%%%% CUSTOM LEGEND SETUP%%%%%%%%%%%%%%%%%%%%
\usepgfplotslibrary{external}
\pgfplotsset{compat=newest}
\newenvironment{customlegend}[1][]{%
	\begingroup
	% inits/clears the lists (which might be populated from previous
	% axes):
	\csname pgfplots@init@cleared@structures\endcsname
	\pgfplotsset{#1}%
}{%
	% draws the legend:
	\csname pgfplots@createlegend\endcsname
	\endgroup
}%

% makes \addlegendimage available (typically only available within an
% axis environment):
\def\addlegendimage{\csname pgfplots@addlegendimage\endcsname}
%%%%%%%%%%%%%%%%%%%%%%%%%%%%%%%%%%%%%%%%%%%%%%%%%%%%%%%%%%%%%%%%%%%%%%%
%%%%%%%%%%%%%%%%%%%%%%%% DEFINE TIKZ LAYERS %%%%%%%%%%%%%%%%%%%%%%%%%%%%
\pgfdeclarelayer{background}
\pgfdeclarelayer{background2}
\pgfdeclarelayer{background1}
\pgfdeclarelayer{foreground1}
\pgfdeclarelayer{foreground2}
\pgfsetlayers{background,background2,background1,main,foreground1,foreground2}

%%%%%%%%%%%% Bibliography setup
\usepackage[
    backend=bibtex,
    style=authoryear, 
    bibencoding=ascii,
    maxbibnames=5,
    sorting=nyt,
    giveninits=true,
    % useprefix=false,
]{biblatex}
\addbibresource{bibliography.bib}

% Customize name format to show only last names and first initials
% \DeclareNameFormat{author}{%
%   \usebibmacro{name:last}{#1}{#3}{#5}{#7}% Last name
%   \ifblank{#3}{}{\addspace\usebibmacro{name:first}{#1}{#3}{#5}{#7}}% First initial
% }

% % Map the `boldnumber` field for conditional bold formatting
% \DeclareSourcemap{
%   \maps[datatype=bibtex]{
%     \map{
%       \step[fieldsource=boldnumber, match=true]
%       \step[fieldset=usera, fieldvalue=true]
%     }
%   }
% }

% % Conditional formatting for bolding only the original number
% \DeclareFieldFormat{labelnumber}{%
%   \ifboolexpr{ test {\iffieldundef{usera}} }
%     {#1}% Regular number if usera is undefined
%     {\textbf{#1}}% Bold number if usera is defined
% }

\setlength\bibitemsep{0.00\baselineskip}