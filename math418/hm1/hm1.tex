\documentclass[hidelinks]{article}[12pt]

\evensidemargin 3mm
\oddsidemargin 3mm
\topmargin -12mm
\textheight 660pt
\textwidth 450pt

% Some basic packages
\usepackage[utf8]{inputenc}
\usepackage[T1]{fontenc}
\usepackage{textcomp}
\usepackage{url}
\usepackage{graphicx}
\usepackage{float}
\usepackage{booktabs}
\usepackage{enumitem}
\usepackage{bbm}

\pdfminorversion=7

% Don't indent paragraphs, leave some space between them
\usepackage{parskip}

% Hide page number when page is empty
\usepackage{emptypage}
\usepackage{subcaption}
\usepackage{multicol}
\usepackage{xcolor}

% Other font I sometimes use.
% \usepackage{cmbright}

% Math stuff
\usepackage{amsmath, amsfonts, mathtools, amsthm, amssymb}
% Fancy script capitals
\usepackage{mathrsfs}
\usepackage{cancel}
% Bold math
\usepackage{bm}
% Some shortcuts
\newcommand\N{\ensuremath{\mathbb{N}}}
\newcommand\R{\ensuremath{\mathbb{R}}}
\newcommand\Z{\ensuremath{\mathbb{Z}}}
\renewcommand\O{\ensuremath{\emptyset}}
\newcommand\Q{\ensuremath{\mathbb{Q}}}
\newcommand\C{\ensuremath{\mathbb{C}}}

\DeclareMathOperator{\var}{Var}
\DeclareMathOperator{\ani}{Ann}
\DeclareMathOperator{\mnn}{Min}
\DeclareMathOperator{\spn}{span}
\DeclareMathOperator{\ch}{ch}
\DeclareMathOperator{\aut}{Aut}
\DeclareMathOperator{\ide}{Id}
\DeclareMathOperator{\iso}{Iso}
\DeclareMathOperator{\sinc}{sinc}


% Easily typeset systems of equations (French package)
\usepackage{systeme}

% Put x \to \infty below \lim
\let\svlim\lim\def\lim{\svlim\limits}

%Make implies and impliedby shorter
\let\implies\Rightarrow
\let\impliedby\Leftarrow
\let\iff\Leftrightarrow
\let\epsilon\varepsilon

% Add \contra symbol to denote contradiction
\usepackage{stmaryrd} % for \lightning
\newcommand\contra{\scalebox{1.5}{$\lightning$}}

% \let\phi\varphi

% Command for short corrections
% Usage: 1+1=\correct{3}{2}

\definecolor{correct}{HTML}{009900}
\newcommand\correct[2]{\ensuremath{\:}{\color{red}{#1}}\ensuremath{\to }{\color{correct}{#2}}\ensuremath{\:}}
\newcommand\green[1]{{\color{correct}{#1}}}

% horizontal rule
\newcommand\hr{
    \noindent\rule[0.5ex]{\linewidth}{0.5pt}
}

% hide parts
\newcommand\hide[1]{}

% si unitx
\usepackage{siunitx}
\sisetup{locale = US}

% Environments
\makeatother
% For box around Definition, Theorem, \ldots
\usepackage{mdframed}
\mdfsetup{skipabove=1em,skipbelow=0em}
\theoremstyle{definition}
\newmdtheoremenv[nobreak=true]{definition}{Definition}
\newmdtheoremenv[nobreak=true]{property}{Property}
\newmdtheoremenv[nobreak=true]{consequence}{Consequence}
\newmdtheoremenv[nobreak=true]{lemma}{Lemma}
\newmdtheoremenv[nobreak=true]{proposition}{Proposition}
\newmdtheoremenv[nobreak=true]{theorem}{Theorem}
\newmdtheoremenv[nobreak=true]{law}{Law}
\newmdtheoremenv[nobreak=true]{corollary}{Corollary}
\newmdtheoremenv{conclusion}{Conclusion}
\newmdtheoremenv{bonus}{Bonus}
\newmdtheoremenv{conjecture}{Conjecture}
\newtheorem*{recall}{Recall}
\newtheorem*{intermezzo}{Intermezzo}
\newtheorem*{notation}{Notation}
\newtheorem*{observation}{Observation}
\newtheorem*{exercise}{Exercise}
\newtheorem*{comment}{Comment}
\newtheorem*{practice}{practice}
\newtheorem*{problem}{Problem}
\newtheorem*{terminology}{Terminology}
\newtheorem*{application}{Application}
\newtheorem*{eg}{Example}
\newtheorem*{question}{Question}
\newtheorem*{previouslyseen}{As previously seen}
\newtheorem*{remark}{Remark}
\newtheorem*{observe}{Observe}
\newtheorem*{intuition}{Intuition}
\newtheorem*{solution}{Solution}

% End example and intermezzo environments with a small diamond (just like proof
% environments end with a small square)
\usepackage{etoolbox}
\AtEndEnvironment{vb}{\null\hfill$\diamond$}%
\AtEndEnvironment{intermezzo}{\null\hfill$\diamond$}%
% \AtEndEnvironment{opmerking}{\null\hfill$\diamond$}%

% Fix some spacing
% http://tex.stackexchange.com/questions/22119/how-can-i-change-the-spacing-before-theorems-with-amsthm
\makeatletter
\def\thm@space@setup{%
  \thm@preskip=\parskip \thm@postskip=0pt
}


% Exercise 
% Usage:
% \oefening{5}
% \suboefening{1}
% \suboefening{2}
% \suboefening{3}
% gives
% Oefening 5
%   Oefening 5.1
%   Oefening 5.2
%   Oefening 5.3
%\newcommand{\exc}[1]{%
    %\def\@excercise{#1}%
    %\subsection*{Excercise #1}
%}

%\newcommand{\sub-excercise}[1]{%
    %\subsubsection*{Excercise \@excercise.#1}
%}


% \lecture starts a new lecture (les in dutch)
%
% Usage:
% \lecture{1}{di 12 feb 2019 16:00}{Inleiding}
%
% This adds a section heading with the number / title of the lecture and a
% margin paragraph with the date.

% I use \dateparts here to hide the year (2019). This way, I can easily parse
% the date of each lecture unambiguously while still having a human-friendly
% short format printed to the pdf.

%\usepackage{xifthen}
%\def\testdateparts#1{\dateparts#1\relax}
%\def\dateparts#1 #2 #3 #4 #5\relax{
    %\marginpar{\small\textsf{\mbox{#1 #2 #3}}}
%}

%\def\@lecture{}%
\newcommand{\note}[3]{
    \ifthenelse{\isempty{#3}}{%
        \def\@note{#1}%
    }{%
        \def\@note{#1: #3}%
    }%
    \subsection*{\@note}
    \marginpar{\small\textsf{\mbox{#2}}}
}



% These are the fancy headers

\makeatother


% Todonotes and inline notes in fancy boxes
\usepackage{tcolorbox}

% Make boxes breakable
\tcbuselibrary{breakable}

% Verbetering is correction in Dutch
% Usage: 
% \begin{verbetering}
%     Lorem ipsum dolor sit amet, consetetur sadipscing elitr, sed diam nonumy eirmod
%     tempor invidunt ut labore et dolore magna aliquyam erat, sed diam voluptua. At
%     vero eos et accusam et justo duo dolores et ea rebum. Stet clita kasd gubergren,
%     no sea takimata sanctus est Lorem ipsum dolor sit amet.
% \end{verbetering}
\newenvironment{correction}{\begin{tcolorbox}[
    arc=0mm,
    colback=white,
    colframe=green!60!black,
    title=Remark,
    fonttitle=\sffamily,
    breakable
]}{\end{tcolorbox}}



% Figure support as explained in my blog post.
\usepackage{import}
\usepackage{xifthen}
\usepackage{pdfpages}
\usepackage{transparent}
\newcommand{\incfig}[1]{%
    \def\svgwidth{\columnwidth}
    \import{./figures/}{#1.pdf_tex}
}

% Fix some stuff
% %http://tex.stackexchange.com/questions/76273/multiple-pdfs-with-page-group-included-in-a-single-page-warning
\pdfsuppresswarningpagegroup=1

% My name
\author{Dominic Klukas}


\usepackage{hyperref}
\usepackage{url}

\newcommand{\Fcal}{{\cal F}}

\begin{document}


\pagestyle{empty}

\noindent
{\bf Math 544   Assignment 1} \hfill  September 8, 2025

\medskip \noindent

Professor: Dr. J. Hermon

Student: Dominic Klukas

Student Number: 64348378

\medskip \noindent
{\bf This assignment is due in Canvas at 23:59 a.m.\ on Friday, September 15. \\
\emph{Late assignments are not accepted.}}



\begin{enumerate}

\item    
In this problem, explain how you are counting---do not just write down an answer without
explanation.
\begin{enumerate}
\item
Compute the probability that a poker hand\footnote{A poker hand
consists of five cards drawn from a deck of 52 cards. The cards have 13 distinct values $2,3,4,5,6,7,8,9,10,J,Q,K,A$, each in four
suits called Hearts, Diamonds, Clubs, Spades.} contains:
\begin{enumerate}
\item
one pair ($aabcd$ with $a,b,c,d$ distinct face values;
answer:  0.4226)
\item
two pairs ($aabbc$ with $a,b,c$ distinct face values;
answer: 0.04754).
\end{enumerate}
\item
Poker dice\footnote{Each die is a cube with six
sides labelled 1,2,3,4,5,6.} is played by simultaneously rolling 5 dice.
Compute the probabilities of the following outcomes:
\begin{enumerate}
\item
one pair ($aabcd$ with $a,b,c,d$ distinct numbers;
answer: 0.4630)
\item
two pairs ($aabbc$ with $a,b,c$ distinct numbers;
answer: 0.2315).
\end{enumerate}
\end{enumerate}

\subsection*{Solution}

\begin{enumerate}
    \item In both cases, we count the number of decks satisfying the required property, and then divide by the total number of decks to get the total probability.
        \begin{enumerate}
            \item First, we determine the number of ways we can count the values. We have 13 choices for the value of $a$, and $\binom{12}{3}$ choices for the values for $b, c, d$.
                Next, we count the number of suits for the different decks. We have $\binom{4}{2}$ choices for the suits of $a$, and 4 choices for each of $b, c, d$.
                Finally, we have $\binom{52}{5}$ decks in total to draw from. Thus, we get:
                \[
                P(\text{One pair}) = \frac{13 \cdot \binom{12}{3} \cdot \binom{4}{2} \cdot 4^3}{\binom{52}{5}} = \frac{1,098,240}{2,598,960} = 0.4226
                .\] 
            \item Similarly, we count values and then suits. 
                For values, we have 13 choices for $c$, and then $\binom{12}{2}$ choices for the values of the pairs $a, a$ and $b, b$. 
                Then, we have 4 choices for the suit of $c$ and $\binom{4}{2}^2$ choices for the suits of $a, a$ and $b, b$.
                In total, this then gives us
                \[
                    P(\text{Two pairs}) = \frac{13 \cdot \binom{12}{2} \cdot 4 \cdot \binom{4}{2}^{2}}{\binom{52}{5}} = \frac{123,552}{2,598,960} = 0.04754
                .\] 
        \end{enumerate}
    \item In this case, we don't have to count suits (since there are only values). For simplicity I will keep track of the orderings of the dice. There are then $6^5$ different dice roll sequences.
        \begin{enumerate}
            \item We have $6 \cdot 5 \cdot 4 \cdot 3$ choices for the numbers, and $\binom{5}{2}$ choices for choosing which indices of the dice in the sequence are doubled. We compute:
                \[
                    P(\text{One pair}) = \frac{6 \cdot 5 \cdot 4 \cdot 3 \cdot \binom{5}{2}}{6^5} = \frac{3600}{7776} = 0.4630
                .\] 
            \item We have $6 \cdot 5 \cdot 4$ choices for the values $a, b, c$. 
                Next, choose which two dice show $a$ ($\binom{5}{2}$ ways), and then which two of the remaining dice show $b$ ($\binom{3}{2}$ ways). 
                Since swapping $a$ and $b$ gives the same outcome, we divide by 2. 
                Finally, the last die is $c$. Thus, we get:
                \[
                P(\text{Two pairs}) = \frac{6 \cdot 5 \cdot 4 \cdot \tfrac{1}{2}\binom{5}{2}\binom{3}{2}}{6^5} = \frac{1800}{7776} = 0.2315
                .\] 
        \end{enumerate}
\end{enumerate}


\item
Let $S=\{1,2,\ldots, n\}$ and suppose that a pair of subsets $(A,B)$ of $S$
is chosen uniformly at random from all possible pairs of subsets.  More precisely, the
probability of choosing any specific pair is $2^{-2n}$.
Show that
\[
    P(A \subset B) = \left( \frac{3}{4} \right)^n.
\]


\subsection*{Solution}

We consider each element $x \in S$ independently. For a uniformly random pair $(A,B)$, the membership of $x$ in $A$ and $B$ is independent and each of the four outcomes
\[
x \in A \cap B,\quad x \notin A \cup B,\quad x \in A \setminus B,\quad x \in B \setminus A
\]
occurs with probability $1/4$.

The event $A \subset B$ fails iff $x \in A \setminus B$. Thus, for a fixed $x$,
\[
P\big(x \in A \Rightarrow x \in B\big)=1-P(x \in A \setminus B)=1-\tfrac14=\tfrac34.
\]
Since the choices for different $x$ are independent, we obtain
\[
P(A \subset B)= \Pi_{x \in S}P(x \in A \implies x \in B) = \left(\tfrac34\right)^n.
\]


\item %Billingsley 2.3 page 33
Let $\Omega$ be a nonempty set and suppose that $\Fcal$ is a collection
of subsets of $\Omega$ such that $\Omega \in \Fcal \subset 2^\Omega$.
\begin{enumerate}
\item
Prove that $\Fcal$ is an algebra if $A,B\in \Fcal$ implies that $A \setminus B = A \cap B^c
\in \Fcal$.
\\
(For an algebra the condition of closure under countable unions in the
definition of a $\sigma$-algebra is replaced by closure under finite unions.)
\item
Suppose that $\Fcal$ is closed under complements and finite \emph{disjoint}
unions.  Show that $\Fcal$
need not be an algebra.
\end{enumerate}
\subsection*{Solution}



\begin{enumerate}
    \item We check the requirements that $\mathcal{F}$ is an algebra.
          \begin{itemize}
            \item $\Omega \in \mathcal{F}$, by assumption.
            \item $\emptyset \in \mathcal{F}$ since $\emptyset = \Omega \setminus \Omega \in \mathcal{F}$.
            \item Closed under complements: for $A \in \mathcal{F}$,
              $A^c = \Omega \setminus A \in \mathcal{F}$.
            \item Closed under finite intersections: for $A,B \in \mathcal{F}$,
              \[
              A \cap B \;=\; A \setminus B^c \in \mathcal{F},
              \]
              since $B^c \in \mathcal{F}$ and $\mathcal{F}$ is closed under set difference.
              By induction, $\bigcap_{i=1}^n A_i \in \mathcal{F}$ for any finite family.
            \item Closed under finite unions: by De Morgan,
              \[
              A \cup B \;=\; (A^c \cap B^c)^c \in \mathcal{F}.
              \]
              Hence $\bigcup_{i=1}^n A_i \in \mathcal{F}$ for any finite family.
          \end{itemize}
          Therefore, $\mathcal{F}$ is an algebra.

    \item Consider the set $\Omega = \left\{ 1, 2, 3, 4 \right\} $, and the family $\mathcal{F} = \left\{ \emptyset, \left\{ 1, 2 \right\} , \left\{ 3, 4 \right\} , \left\{ 1, 4 \right\} , \left\{ 2, 3 \right\} , \left\{  1, 2, 3, 4 \right\} \right\} $.
        Aside from pairs containing the empty set, the only other disjoint unions are $\left\{ 1, 2 \right\} \dot\cup \left\{ 3, 4 \right\} = \Omega$ and $\left\{ 1, 4 \right\} \dot\cup \left\{ 2, 3 \right\} = \Omega$.
        Also, $\mathcal{F}$ is closed under complements: Indeed, we have, for non-empty sets, the following complementary pairs: $\left\{ 1, 2 \right\} = \left\{3, 4  \right\}^{c} $, and $\left\{ 1, 4 \right\}  = \left\{ 2, 3 \right\}^{c}$.
        Therefore, $\mathcal{F}$ satisfies the requirements of this definition.
        However, $\mathcal{F}$ is not closed under finite intersections, so it is not an algebra, since $\left\{ 1, 2 \right\} \cap \left\{ 2, 3 \right\} = \left\{ 2 \right\}  \not\in  \mathcal{F}$.
    \item
        Consider $\Omega = \{1,2,3,4\}$ and
        \[
        \mathcal{F} = \{\emptyset,\ \{1,2\},\ \{3,4\},\ \{1,4\},\ \{2,3\},\ \Omega\}.
        \]
        Then $\mathcal{F}$ is closed under complements (e.g., $\{1,2\}^c=\{3,4\}$ and $\{1,4\}^c=\{2,3\}$).
        It is also closed under finite disjoint unions: aside from unions with $\emptyset$, the only nontrivial disjoint pairs are
        $\{1,2\}\dot\cup\{3,4\}=\Omega$ and $\{1,4\}\dot\cup\{2,3\}=\Omega$, both in $\mathcal{F}$.

        However, $\mathcal{F}$ is not an algebra because it is not closed under finite intersections or finite unions. We can see that:
        \[
        \{1,2\}\cap\{2,3\}=\{2\}\notin\mathcal{F}
        \quad\text{and}\quad
        \{1,2\}\cup\{2,3\}=\{1,2,3\}\notin\mathcal{F}.
        \]
        Thus, closure under complements and finite disjoint unions does not imply that $\mathcal{F}$ is an algebra.

\end{enumerate}

\item
\begin{enumerate}
\item
Given an arbitrary nonempty
collection of sets $\{E_\alpha : \alpha \in A\}$, prove that there is
a smallest $\sigma$-algebra that contains every $E_\alpha$.  This $\sigma$-algebra
is called the $\sigma$-algebra \emph{generated} by $\{E_\alpha : \alpha \in A\}$.

\item
Suppose that we have $\sigma$-algebras $\Fcal_1$ and $\Fcal_2$.
Show (by counterexample) that the union $\Fcal_1 \cup \Fcal_2$
need not be a $\sigma$-algebra.
\end{enumerate}

\subsection*{Solution}

\begin{enumerate}
    \item Let $\Omega = \bigcup_{\alpha \in A} E_\alpha$ and set $S = \{E_\alpha : \alpha \in A\}$.
    There exists at least one $\sigma$-algebra on $\Omega$ containing $S$, namely $2^\Omega$.

    Define
    \[
      \mathcal{F} \;=\; \bigcap_{\substack{\mathcal{X}\ \text{a }\sigma\text{-algebra on }\Omega\\ S \subset \mathcal{X}}} \mathcal{X}.
    \]
    We check that $\mathcal{F}$ is a $\sigma$-algebra: clearly $\Omega,\emptyset \in \mathcal{F}$, since every $\sigma$-algebra on $\Omega$ must contain $\emptyset$ and $\Omega$.
    If $A \in \mathcal{F}$, then $A \in \mathcal{X}$ for every such $\mathcal{X}$, hence $A^c \in \mathcal{X}$ for each $\mathcal{X}$, so $A^c \in \mathcal{F}$.
    If $A_1,A_2,\ldots \in \mathcal{F}$, then $A_i \in \mathcal{X}$ for every $\mathcal{X}$, and since each $\mathcal{X}$ is a $\sigma$-algebra, $\bigcup_{i=1}^\infty A_i \in \mathcal{X}$ for all $\mathcal{X}$; hence $\bigcup_{i=1}^\infty A_i \in \mathcal{F}$.
    Thus $\mathcal{F}$ is a $\sigma$-algebra containing $S$.
    Finally, if $\mathcal{G}$ is any $\sigma$-algebra on $\Omega$ with $S \subset \mathcal{G}$, then $\mathcal{G}$ appears in the intersection above, so $\mathcal{F} \subset \mathcal{G}$.
    Therefore $\mathcal{F}$ is the smallest $\sigma$-algebra containing $\{E_\alpha : \alpha \in A\}$ (the $\sigma$-algebra generated by this collection).

    \item If the whole space isn’t fixed, we could take $\mathcal{F}_1 = \{\emptyset,\{1\}\}$ and $\mathcal{F}_2=\{\emptyset,\{2\}\}$; then $\mathcal{F}_1\cup\mathcal{F}_2=\{\emptyset,\{1\},\{2\}\}$ is not a $\sigma$-algebra (e.g., it is not closed under unions, and doesn't contain the whole space).

    If we insist both $\sigma$-algebras are on the same sample space, take $\Omega=\{1,2,3\}$ and
    \[
      \mathcal{F}_1=\{\emptyset,\{1\},\{2,3\},\Omega\},\qquad
      \mathcal{F}_2=\{\emptyset,\{1,2\},\{3\},\Omega\}.
    \]
    Both $\mathcal{F}_1$ and $\mathcal{F}_2$ are $\sigma$-algebras on $\Omega$, but
    \[
      \mathcal{F}_1 \cup \mathcal{F}_2
      = \{\emptyset,\{1\},\{2,3\},\{1,2\},\{3\},\Omega\}
    \]
    is not a $\sigma$-algebra: for example,
    $\{1,2\} \cap \{2,3\} = \{2\} \notin \mathcal{F}_1 \cup \mathcal{F}_2$.
    Hence the union of two $\sigma$-algebras need not be a $\sigma$-algebra.
\end{enumerate}

\item
Let $\Fcal$ be the $\sigma$-algebra generated by an arbitrary nonempty
collection of sets $\{E_\alpha : \alpha \in A\}$.  Prove that for each $E \in
\Fcal$, there exists a countable subcollection $\{E_{\alpha_j}: j=1,2,\ldots\}$
(depending on $E$) such that $E$ already belongs to the $\sigma$-algebra
generated by this subcollection.
\\
Hint: Consider the class of all sets with the asserted property and
show that it is a $\sigma$-algebra containing each $E_\alpha$.

\subsection*{Solution}
We follow the hint. Let $\mathcal{C}$ be the collection of all sets $E$ such that there exists a countable subcollection $\{E_{\alpha_j} : j=1,2,\ldots\}$ with $E$ belonging to the $\sigma$-algebra generated by this subcollection.

Clearly, every $E_\alpha$ itself belongs to $\mathcal{C}$, since $E_\alpha$ is in the $\sigma$-algebra generated by $\{E_\alpha\}$.

Now we check that $\mathcal{C}$ is a $\sigma$-algebra:

\begin{itemize}
  \item $\Omega,\emptyset \in \mathcal{C}$ provided that $\left\{ E_{\alpha} : \alpha \in A \right\} $ is non-empty, since $\emptyset, \Omega$ are present in every $\sigma$-algebra including those generated by countably many sets.

  \item Closed under complements: if $E \in \mathcal{C}$, then there is a countable subcollection $B$ with $E \in \mathcal{F}_B$, where $\mathcal{F}_B$ is the $\sigma$-algebra generated by $B$. Since $\mathcal{F}_B$ is a $\sigma$-algebra, $E^c \in \mathcal{F}_B$, hence $E^c \in \mathcal{C}$.

  \item Closed under countable unions: if $E_1,E_2,\ldots \in \mathcal{C}$, then for each $i$ there is a countable $B_i$ with $E_i \in \mathcal{F}_{B_i}$. The union $B = \bigcup_i B_i$ is countable, and $\mathcal{F}_B$ is a $\sigma$-algebra containing each $E_i$, hence also $\bigcup_i E_i$. Thus $\bigcup_i E_i \in \mathcal{C}$.
\end{itemize}

Therefore, $\mathcal{C}$ is a $\sigma$-algebra containing all $E_\alpha$. By definition, the $\sigma$-algebra generated by $\{E_\alpha : \alpha \in A\}$ is the smallest such, so
\[
\mathcal{F} \subset \mathcal{C}.
\]
Thus, for each $E \in \mathcal{F}$ there exists a countable subcollection $\{E_{\alpha_j}\}$ such that $E$ lies in the $\sigma$-algebra generated by that subcollection.




\end{enumerate}

\bigskip   \noindent
{\bf Recommended problems.}
Each assignment will include additional recommended
problems, which are not to be handed in for marking.
The following problems from Rosenthal are recommended but are not to be handed in:

2.7.2, 2.7.4, 2.7.6.
\\
For solutions to even-numbered problems see: \url{http://www.probability.ca/jeff/grprobbook.html}.

\bigskip \noindent
Quote of the week: {\em
The student of arithmetic who has mastered the first four rules of his art,
and successfully striven with money sums and fractions, finds himself
confronted by an unbroken expanse of questions known as problems.
These are short stories of adventure and industry with the end omitted,
and though betraying a strong family resemblance, are not without a certain
element of romance.
}

\hspace*{\fill} Stephen Leacock


\end{document}
