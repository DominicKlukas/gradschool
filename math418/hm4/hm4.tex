\documentclass[10pt]{article}
\usepackage[a4paper,margin=1in]{geometry}
\usepackage{amsmath,amssymb,amsthm}
\usepackage{parskip}

% Some basic packages
\usepackage[utf8]{inputenc}
\usepackage[T1]{fontenc}
\usepackage{textcomp}
\usepackage{url}
\usepackage{graphicx}
\usepackage{float}
\usepackage{booktabs}
\usepackage{enumitem}
\usepackage{bbm}

\pdfminorversion=7

% Don't indent paragraphs, leave some space between them
\usepackage{parskip}

% Hide page number when page is empty
\usepackage{emptypage}
\usepackage{subcaption}
\usepackage{multicol}
\usepackage{xcolor}

% Other font I sometimes use.
% \usepackage{cmbright}

% Math stuff
\usepackage{amsmath, amsfonts, mathtools, amsthm, amssymb}
% Fancy script capitals
\usepackage{mathrsfs}
\usepackage{cancel}
% Bold math
\usepackage{bm}
% Some shortcuts
\newcommand\N{\ensuremath{\mathbb{N}}}
\newcommand\R{\ensuremath{\mathbb{R}}}
\newcommand\Z{\ensuremath{\mathbb{Z}}}
\renewcommand\O{\ensuremath{\emptyset}}
\newcommand\Q{\ensuremath{\mathbb{Q}}}
\newcommand\C{\ensuremath{\mathbb{C}}}

\DeclareMathOperator{\var}{Var}
\DeclareMathOperator{\ani}{Ann}
\DeclareMathOperator{\mnn}{Min}
\DeclareMathOperator{\spn}{span}
\DeclareMathOperator{\ch}{ch}
\DeclareMathOperator{\aut}{Aut}
\DeclareMathOperator{\ide}{Id}
\DeclareMathOperator{\iso}{Iso}
\DeclareMathOperator{\sinc}{sinc}


% Easily typeset systems of equations (French package)
\usepackage{systeme}

% Put x \to \infty below \lim
\let\svlim\lim\def\lim{\svlim\limits}

%Make implies and impliedby shorter
\let\implies\Rightarrow
\let\impliedby\Leftarrow
\let\iff\Leftrightarrow
\let\epsilon\varepsilon

% Add \contra symbol to denote contradiction
\usepackage{stmaryrd} % for \lightning
\newcommand\contra{\scalebox{1.5}{$\lightning$}}

% \let\phi\varphi

% Command for short corrections
% Usage: 1+1=\correct{3}{2}

\definecolor{correct}{HTML}{009900}
\newcommand\correct[2]{\ensuremath{\:}{\color{red}{#1}}\ensuremath{\to }{\color{correct}{#2}}\ensuremath{\:}}
\newcommand\green[1]{{\color{correct}{#1}}}

% horizontal rule
\newcommand\hr{
    \noindent\rule[0.5ex]{\linewidth}{0.5pt}
}

% hide parts
\newcommand\hide[1]{}

% si unitx
\usepackage{siunitx}
\sisetup{locale = US}

% Environments
\makeatother
% For box around Definition, Theorem, \ldots
\usepackage{mdframed}
\mdfsetup{skipabove=1em,skipbelow=0em}
\theoremstyle{definition}
\newmdtheoremenv[nobreak=true]{definition}{Definition}
\newmdtheoremenv[nobreak=true]{property}{Property}
\newmdtheoremenv[nobreak=true]{consequence}{Consequence}
\newmdtheoremenv[nobreak=true]{lemma}{Lemma}
\newmdtheoremenv[nobreak=true]{proposition}{Proposition}
\newmdtheoremenv[nobreak=true]{theorem}{Theorem}
\newmdtheoremenv[nobreak=true]{law}{Law}
\newmdtheoremenv[nobreak=true]{corollary}{Corollary}
\newmdtheoremenv{conclusion}{Conclusion}
\newmdtheoremenv{bonus}{Bonus}
\newmdtheoremenv{conjecture}{Conjecture}
\newtheorem*{recall}{Recall}
\newtheorem*{intermezzo}{Intermezzo}
\newtheorem*{notation}{Notation}
\newtheorem*{observation}{Observation}
\newtheorem*{exercise}{Exercise}
\newtheorem*{comment}{Comment}
\newtheorem*{practice}{practice}
\newtheorem*{problem}{Problem}
\newtheorem*{terminology}{Terminology}
\newtheorem*{application}{Application}
\newtheorem*{eg}{Example}
\newtheorem*{question}{Question}
\newtheorem*{previouslyseen}{As previously seen}
\newtheorem*{remark}{Remark}
\newtheorem*{observe}{Observe}
\newtheorem*{intuition}{Intuition}
\newtheorem*{solution}{Solution}

% End example and intermezzo environments with a small diamond (just like proof
% environments end with a small square)
\usepackage{etoolbox}
\AtEndEnvironment{vb}{\null\hfill$\diamond$}%
\AtEndEnvironment{intermezzo}{\null\hfill$\diamond$}%
% \AtEndEnvironment{opmerking}{\null\hfill$\diamond$}%

% Fix some spacing
% http://tex.stackexchange.com/questions/22119/how-can-i-change-the-spacing-before-theorems-with-amsthm
\makeatletter
\def\thm@space@setup{%
  \thm@preskip=\parskip \thm@postskip=0pt
}


% Exercise 
% Usage:
% \oefening{5}
% \suboefening{1}
% \suboefening{2}
% \suboefening{3}
% gives
% Oefening 5
%   Oefening 5.1
%   Oefening 5.2
%   Oefening 5.3
%\newcommand{\exc}[1]{%
    %\def\@excercise{#1}%
    %\subsection*{Excercise #1}
%}

%\newcommand{\sub-excercise}[1]{%
    %\subsubsection*{Excercise \@excercise.#1}
%}


% \lecture starts a new lecture (les in dutch)
%
% Usage:
% \lecture{1}{di 12 feb 2019 16:00}{Inleiding}
%
% This adds a section heading with the number / title of the lecture and a
% margin paragraph with the date.

% I use \dateparts here to hide the year (2019). This way, I can easily parse
% the date of each lecture unambiguously while still having a human-friendly
% short format printed to the pdf.

%\usepackage{xifthen}
%\def\testdateparts#1{\dateparts#1\relax}
%\def\dateparts#1 #2 #3 #4 #5\relax{
    %\marginpar{\small\textsf{\mbox{#1 #2 #3}}}
%}

%\def\@lecture{}%
\newcommand{\note}[3]{
    \ifthenelse{\isempty{#3}}{%
        \def\@note{#1}%
    }{%
        \def\@note{#1: #3}%
    }%
    \subsection*{\@note}
    \marginpar{\small\textsf{\mbox{#2}}}
}



% These are the fancy headers

\makeatother


% Todonotes and inline notes in fancy boxes
\usepackage{tcolorbox}

% Make boxes breakable
\tcbuselibrary{breakable}

% Verbetering is correction in Dutch
% Usage: 
% \begin{verbetering}
%     Lorem ipsum dolor sit amet, consetetur sadipscing elitr, sed diam nonumy eirmod
%     tempor invidunt ut labore et dolore magna aliquyam erat, sed diam voluptua. At
%     vero eos et accusam et justo duo dolores et ea rebum. Stet clita kasd gubergren,
%     no sea takimata sanctus est Lorem ipsum dolor sit amet.
% \end{verbetering}
\newenvironment{correction}{\begin{tcolorbox}[
    arc=0mm,
    colback=white,
    colframe=green!60!black,
    title=Remark,
    fonttitle=\sffamily,
    breakable
]}{\end{tcolorbox}}



% Figure support as explained in my blog post.
\usepackage{import}
\usepackage{xifthen}
\usepackage{pdfpages}
\usepackage{transparent}
\newcommand{\incfig}[1]{%
    \def\svgwidth{\columnwidth}
    \import{./figures/}{#1.pdf_tex}
}

% Fix some stuff
% %http://tex.stackexchange.com/questions/76273/multiple-pdfs-with-page-group-included-in-a-single-page-warning
\pdfsuppresswarningpagegroup=1

% My name
\author{Dominic Klukas}


\begin{document}

% ==== BEGIN ASSIGNMENT BODY (paste after \begin{document}) ====

 \textbf{Math 418/544 Assignment 4} 
 \hfill September 29, 2025\\[6pt]
    Dr. J. Hermon

\vspace{0.5em}
\textbf{This assignment is due in Canvas at 23:59 p.m. on Monday, October 6.}\\
\textbf{\textit{Late assignments are not accepted.}}

\vspace{0.5em}
\begin{enumerate}[leftmargin=1.2cm]

  \item A searchlight is distance 1 from an infinitely long wall. Let $Q$ denote the closest point on the wall
  and assume the searchlight scans along the wall so that at any given time, the angle $\Theta$ the beam of
  light makes is uniform on $(-\pi/2, \pi/2)$ (the point $Q$ corresponds to angle 0). Let $X \in \mathbb{R}$ (positive
  or negative) be the position of the beam on the wall as measured from $Q$. Find the cumulative
  distribution function and density function for $X$.  
  \emph{(This is the Cauchy distribution; its expectation is undefined.)}

  \textbf{Solution}
  
  If the angle sampled from the uniform distribution $\omega \in (-\pi / 2, \pi / 2)$, then we have that the point on the wall is given by $z = \tan(\omega) \times   (1 \text{m})$.
  Thus, the random variable $Z(\omega) = \tan(\omega)$.
  To compute the CDF, we notice that the uniform distribution over $(-\pi / 2, \pi / 2)$ is $\frac{1}{\pi} \lambda$: the lebesgue measure normalized on $(-\pi / 2, \pi / 2)$.
  We then compute:
  \begin{align*}
      F_Z(z) = P(Z(\omega) \le  z) &= P(\omega \le  \tan^{-1}(z))  \\
      &=\frac{1}{\pi} \lambda((-\pi / 2, \tan^{-1}(z))) \\
      &= \frac{1}{\pi} \tan^{-1}(z) + \frac{1}{2}
  .\end{align*}

  The probability density function, by FTC, is given by
  \[
      f_Z(z) = F_Z'(z) = \frac{1}{\pi} \frac{1}{x^2 + 1}
  .\]

  \item Suppose that $X, Y$ are independent standard normal random variables.\footnote{The independence has the consequence that the joint density $g$ of $X, Y$ is the product of the densities. In other words, for a Borel subset $B$ of $\mathbb{R}^2$,
  \[
  P((X,Y) \in B) = \int_B g(x,y)\, dxdy
  \]
  with $g(x,y) = f_X(x)f_Y(y)$. We have not yet discussed integration over arbitrary Borel sets (this requires Lebesgue
  integration which we will discuss soon) but the sets $B$ encountered in this assignment are nice enough that Riemann
  integration does the job.}  
  By differentiating the cumulative distribution function for the random variable $Z = X/Y$ to obtain the density of $Z$,
  show that $Z$ has a standard Cauchy distribution.

  \textbf{Solution.}

  We follow the hint in the footnote. 
  Now, $F_Z(z) = P(\frac{X}{Y} \le  z)$ is described by $x, y \in \R^2$, where $X(\omega) = x$ and $Y(\omega) = y$ such that $\frac{x}{y} \le  z$.
  This set, is then described by
  \[
  B = (\{X \le  z Y\}  \cap \{Y >  0\} ) \cup (\{X \ge  z Y \} \cap \{Y < 0\}  )
  .\] 
  We integrate over this region, with the pdf being $f_X(x)\cdot f_Y(y)$.
  When we differentiate, we use the Leibnitz integral rule:
  \begin{align*}
      F_Z(z) &= \int_{-\infty}^{0} \int_{yz}^{\infty} f_X(x) f_Y(y) dx dy + \int_{0}^{\infty}\int_{-\infty}^{yz} f_X(x)f_Y(y) dx dy \\
     f_Z(z) = F'(z) &= \int_{-\infty}^{0} -y f_X(yz)f_Y(y) dy + \int_{0}^{\infty} f_X(yz) f_Y(y) dx dy \\
     &= 5
  .\end{align*}

  Now, we substitue the pdf's for the normal distributions and compute each of these integrals:
  \begin{align*}
      \frac{1}{2\pi} \int_{-\infty}^{0} (-y) \exp(-\frac{1}{2}(z^2 + 1) y^2)dy &=  \frac{1}{2\pi} \int_{\infty}^{0} (- 1 / 2) \exp(- \frac{1}{2} (z^2 + 1) u) du \\
      &= \frac{1}{2\pi} (- \frac{1}{2})(\frac{-2}{z^2 + 1}) \left[ \exp(- \frac{1}{2} (z^2 + 1)) \right]_{\infty}^{0}  \\
      &= \frac{1}{2\pi (z^2 + 1)}
  .\end{align*}
  \begin{align*}
      \frac{1}{2\pi} \int_{0}^{\infty} y \exp(-\frac{1}{2}(z^2 + 1) y^2)dy &=  \frac{1}{2\pi} \int_{0}^{\infty} ( 1 / 2) \exp(- \frac{1}{2} (z^2 + 1) u) du \\
      &= \frac{1}{2\pi} (\frac{1}{2})(\frac{-2}{z^2 + 1}) \left[ \exp(- \frac{1}{2} (z^2 + 1)) \right]_{0}^{\infty}  \\
      &= \frac{1}{2\pi (z^2 + 1)}(0 - 1) = \frac{1}{2\pi (z^2 + 1)}
  .\end{align*}
  Thus,
  \[
  f_Z(z) = \frac{1}{2\pi (z^2 + 1)} + \frac{1}{2\pi (z^2 + 1)} = \frac{1}{\pi (z^2 + 1)}
  .\] 

  \item Suppose that $X, Y$ are independent $\mathrm{Exp}(\lambda)$ random variables. By differentiating the cumulative
  distribution function for the random variable $Z = X - Y$, find the density function of $Z$ (it is a Laplace distribution).

  Similar to last problem, we fix $z$ and determine the probability $P(Z \le  z)$.
  We consider two cases for $z$, $z \ge  0$ and $z < 0$.
  First, consider $z > 0$.
  In particular, this occurs when $X \le  Y + z$.
  Also, we recall that $X \ge  0$ and $Y \ge  0$ (since they are independent exponential random variables).
  We integrate over this region with the pdf being the joint pdf of $X$ and $Y$.

  \begin{align*}
      F_Z(z) &= \int_{0}^{\infty} \int_{0}^{y + z} f_X(x) f_Y(y) dx dy\\
      f_Z(z) = F_Z'(z)&= \int_{0}^{\infty} f_X(y + z) f_Y(y) dy
  .\end{align*}

  If $z \le  0$, then instead, we can use the equivalent inequality $X + (-z) \le  Y$, where $-z > 0$.
  Then,
  \begin{align*}
      F_Z(z) &= \int_{0}^{\infty} \int_{0}^{x - z} f_X(x) f_Y(y) dy dx\\
      f_Z(z) = F'(z) &= \int_{0}^{\infty} f_X(x) f_Y(x-z) dx 
  .\end{align*}

  We compute each of these integrals.
  For $z > 0$:
  \begin{align*}
      \int_{0}^{\infty} f_X(y + z) f_Y(y) dy &= \int_{0}^{\infty} \lambda^2 \exp(-\lambda(y + z)) \exp(-\lambda y)dy \\
      &= \lambda^2 \exp(-\lambda z) \int_{0}^{\infty} \exp(-2 \lambda y) dy \\
      &= \lambda^2 \exp(-\lambda z) \frac{1}{2 \lambda} = \frac{\lambda}{2} \exp(- \lambda z)
  .\end{align*}
  Likewise, for $z < 0$, we compute a very similar integral to be $\frac{\lambda}{2} \exp(\lambda z)$.
  Putting these together, we get $f_Z(z) = F_Z'(z) = \frac{\lambda}{2} \exp(-\lambda |z|)$, as desired.

  \item Suppose that $X$ is a random variable whose cumulative distribution function $F$ is continuous on
  $\mathbb{R}$. Let $Y = F(X)$. Prove that $Y$ has a uniform distribution, i.e., that $F_Y(x) = x$ for $x \in (0,1)$.

  \item Let $X_0, X_1, \ldots$ be i.i.d.\ continuous random variables
  with cumulative distribution function
  \[
  F(x) = \int_{-\infty}^x f(t)\, dt,
  \]
  for some probability density function $f : \mathbb{R} \to [0,\infty)$. Let
  \[
  N = \min\{n : X_n > X_0\}.
  \]
  \begin{enumerate}
    \item Show that the cumulative distribution function $F_N$ of $N$ is given by
    \[
    F_N(n) = 1 - \frac{1}{n+1}
    \]
    for $n \in \mathbb{N}$. (Of course $F_N(x) = 0$ for $x < 1$ and $F_N(x) = F_N(n)$ for $x \in (n, n+1)$.)
    
    \item Show\footnote{One way, not the only way, is to employ a higher-dimensional version of the previous footnote.} 
    that the cumulative distribution function of $X_N$ is
    \[
    F + (1-F)\log(1-F).
    \]
    (Recall that $\log(1 - t) = -\sum_{n=1}^\infty \tfrac{t^n}{n}$ for $|t| < 1$; if $F = 1$ we interpret $(1-F)\log(1-F)$ as
    zero.)
  \end{enumerate}
\end{enumerate}

\textbf{Recommended problems (Do not hand in).}
\begin{enumerate}
  \item Suppose that $X$ has a continuous density function $f$, that $P(\alpha \leq X \leq \beta) = 1$, and that
  $g$ is a strictly increasing differentiable function on $(\alpha,\beta)$. Show that the density of $g(X)$ is
  $f(g^{-1}(x)) / g'(g^{-1}(x))$ if $x \in (g(\alpha), g(\beta))$ and otherwise is zero.

  \item Use the result of (1) to determine the density of $aX + b$ for constants $a > 0$ and $b \in \mathbb{R}$.

  \item Suppose $X$ has a standard normal distribution. Use the result of (1) to compute the density of
  $\exp(X)$ (this is the lognormal distribution).

  \item Suppose that $X$ has continuous density function $f$. Find the density of $X^2$. Let $Z$ have a standard
  normal distribution, and find the density of $Z^2$ (this is a special case of the $\chi^2$-distribution).
\end{enumerate}

\begin{quote}
\emph{Quote of the week: Probability theory is measure theory with a soul.}  
--- Mark Kac
\end{quote}


\end{document}
