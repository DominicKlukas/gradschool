\documentclass[hidelinks]{article}[12pt]

\oddsidemargin 3mm
\evensidemargin 3mm
\topmargin -12mm
\textheight 630pt
\textwidth 450pt

% Some basic packages
\usepackage[utf8]{inputenc}
\usepackage[T1]{fontenc}
\usepackage{textcomp}
\usepackage{url}
\usepackage{graphicx}
\usepackage{float}
\usepackage{booktabs}
\usepackage{enumitem}
\usepackage{bbm}

\pdfminorversion=7

% Don't indent paragraphs, leave some space between them
\usepackage{parskip}

% Hide page number when page is empty
\usepackage{emptypage}
\usepackage{subcaption}
\usepackage{multicol}
\usepackage{xcolor}

% Other font I sometimes use.
% \usepackage{cmbright}

% Math stuff
\usepackage{amsmath, amsfonts, mathtools, amsthm, amssymb}
% Fancy script capitals
\usepackage{mathrsfs}
\usepackage{cancel}
% Bold math
\usepackage{bm}
% Some shortcuts
\newcommand\N{\ensuremath{\mathbb{N}}}
\newcommand\R{\ensuremath{\mathbb{R}}}
\newcommand\Z{\ensuremath{\mathbb{Z}}}
\renewcommand\O{\ensuremath{\emptyset}}
\newcommand\Q{\ensuremath{\mathbb{Q}}}
\newcommand\C{\ensuremath{\mathbb{C}}}

\DeclareMathOperator{\var}{Var}
\DeclareMathOperator{\ani}{Ann}
\DeclareMathOperator{\mnn}{Min}
\DeclareMathOperator{\spn}{span}
\DeclareMathOperator{\ch}{ch}
\DeclareMathOperator{\aut}{Aut}
\DeclareMathOperator{\ide}{Id}
\DeclareMathOperator{\iso}{Iso}
\DeclareMathOperator{\sinc}{sinc}


% Easily typeset systems of equations (French package)
\usepackage{systeme}

% Put x \to \infty below \lim
\let\svlim\lim\def\lim{\svlim\limits}

%Make implies and impliedby shorter
\let\implies\Rightarrow
\let\impliedby\Leftarrow
\let\iff\Leftrightarrow
\let\epsilon\varepsilon

% Add \contra symbol to denote contradiction
\usepackage{stmaryrd} % for \lightning
\newcommand\contra{\scalebox{1.5}{$\lightning$}}

% \let\phi\varphi

% Command for short corrections
% Usage: 1+1=\correct{3}{2}

\definecolor{correct}{HTML}{009900}
\newcommand\correct[2]{\ensuremath{\:}{\color{red}{#1}}\ensuremath{\to }{\color{correct}{#2}}\ensuremath{\:}}
\newcommand\green[1]{{\color{correct}{#1}}}

% horizontal rule
\newcommand\hr{
    \noindent\rule[0.5ex]{\linewidth}{0.5pt}
}

% hide parts
\newcommand\hide[1]{}

% si unitx
\usepackage{siunitx}
\sisetup{locale = US}

% Environments
\makeatother
% For box around Definition, Theorem, \ldots
\usepackage{mdframed}
\mdfsetup{skipabove=1em,skipbelow=0em}
\theoremstyle{definition}
\newmdtheoremenv[nobreak=true]{definition}{Definition}
\newmdtheoremenv[nobreak=true]{property}{Property}
\newmdtheoremenv[nobreak=true]{consequence}{Consequence}
\newmdtheoremenv[nobreak=true]{lemma}{Lemma}
\newmdtheoremenv[nobreak=true]{proposition}{Proposition}
\newmdtheoremenv[nobreak=true]{theorem}{Theorem}
\newmdtheoremenv[nobreak=true]{law}{Law}
\newmdtheoremenv[nobreak=true]{corollary}{Corollary}
\newmdtheoremenv{conclusion}{Conclusion}
\newmdtheoremenv{bonus}{Bonus}
\newmdtheoremenv{conjecture}{Conjecture}
\newtheorem*{recall}{Recall}
\newtheorem*{intermezzo}{Intermezzo}
\newtheorem*{notation}{Notation}
\newtheorem*{observation}{Observation}
\newtheorem*{exercise}{Exercise}
\newtheorem*{comment}{Comment}
\newtheorem*{practice}{practice}
\newtheorem*{problem}{Problem}
\newtheorem*{terminology}{Terminology}
\newtheorem*{application}{Application}
\newtheorem*{eg}{Example}
\newtheorem*{question}{Question}
\newtheorem*{previouslyseen}{As previously seen}
\newtheorem*{remark}{Remark}
\newtheorem*{observe}{Observe}
\newtheorem*{intuition}{Intuition}
\newtheorem*{solution}{Solution}

% End example and intermezzo environments with a small diamond (just like proof
% environments end with a small square)
\usepackage{etoolbox}
\AtEndEnvironment{vb}{\null\hfill$\diamond$}%
\AtEndEnvironment{intermezzo}{\null\hfill$\diamond$}%
% \AtEndEnvironment{opmerking}{\null\hfill$\diamond$}%

% Fix some spacing
% http://tex.stackexchange.com/questions/22119/how-can-i-change-the-spacing-before-theorems-with-amsthm
\makeatletter
\def\thm@space@setup{%
  \thm@preskip=\parskip \thm@postskip=0pt
}


% Exercise 
% Usage:
% \oefening{5}
% \suboefening{1}
% \suboefening{2}
% \suboefening{3}
% gives
% Oefening 5
%   Oefening 5.1
%   Oefening 5.2
%   Oefening 5.3
%\newcommand{\exc}[1]{%
    %\def\@excercise{#1}%
    %\subsection*{Excercise #1}
%}

%\newcommand{\sub-excercise}[1]{%
    %\subsubsection*{Excercise \@excercise.#1}
%}


% \lecture starts a new lecture (les in dutch)
%
% Usage:
% \lecture{1}{di 12 feb 2019 16:00}{Inleiding}
%
% This adds a section heading with the number / title of the lecture and a
% margin paragraph with the date.

% I use \dateparts here to hide the year (2019). This way, I can easily parse
% the date of each lecture unambiguously while still having a human-friendly
% short format printed to the pdf.

%\usepackage{xifthen}
%\def\testdateparts#1{\dateparts#1\relax}
%\def\dateparts#1 #2 #3 #4 #5\relax{
    %\marginpar{\small\textsf{\mbox{#1 #2 #3}}}
%}

%\def\@lecture{}%
\newcommand{\note}[3]{
    \ifthenelse{\isempty{#3}}{%
        \def\@note{#1}%
    }{%
        \def\@note{#1: #3}%
    }%
    \subsection*{\@note}
    \marginpar{\small\textsf{\mbox{#2}}}
}



% These are the fancy headers

\makeatother


% Todonotes and inline notes in fancy boxes
\usepackage{tcolorbox}

% Make boxes breakable
\tcbuselibrary{breakable}

% Verbetering is correction in Dutch
% Usage: 
% \begin{verbetering}
%     Lorem ipsum dolor sit amet, consetetur sadipscing elitr, sed diam nonumy eirmod
%     tempor invidunt ut labore et dolore magna aliquyam erat, sed diam voluptua. At
%     vero eos et accusam et justo duo dolores et ea rebum. Stet clita kasd gubergren,
%     no sea takimata sanctus est Lorem ipsum dolor sit amet.
% \end{verbetering}
\newenvironment{correction}{\begin{tcolorbox}[
    arc=0mm,
    colback=white,
    colframe=green!60!black,
    title=Remark,
    fonttitle=\sffamily,
    breakable
]}{\end{tcolorbox}}



% Figure support as explained in my blog post.
\usepackage{import}
\usepackage{xifthen}
\usepackage{pdfpages}
\usepackage{transparent}
\newcommand{\incfig}[1]{%
    \def\svgwidth{\columnwidth}
    \import{./figures/}{#1.pdf_tex}
}

% Fix some stuff
% %http://tex.stackexchange.com/questions/76273/multiple-pdfs-with-page-group-included-in-a-single-page-warning
\pdfsuppresswarningpagegroup=1

% My name
\author{Dominic Klukas}



\usepackage{hyperref}
\usepackage{url}
\usepackage{amstext}
\usepackage{amsmath,amssymb}
\usepackage{graphicx}
\usepackage{bbm}

\newcommand{\Fcal}{{\cal F}}
\newcommand{\1}{\mathbbm{1}}


\begin{document}


%\pagestyle{empty}

\noindent
{\bf Math 418/544   Assignment 5} \hfill  October 10, 2025

\bigskip \noindent
Dr.\ J.\ Hermon

\bigskip \noindent
{\bf This assignment is due in Canvas at 11:59 p.m.\ on Friday, October 17. \\
\emph{Late assignments are not accepted.}}


Dominic Klukas

SN 64348378

\begin{enumerate}



\item
%Grimmett and Stirzaker 7.3.10.
Let $X_1,X_2,\ldots$ be independent $N(0,1)$ (standard normal) random variables.  Prove that
\[
    P\Big(\limsup_{n\to\infty} \frac{|X_n|}{\sqrt{\log n}} = \sqrt{2}\Big) = 1.
\]
(You may use the fact that the cumulative distribution function $\Phi$ of
the standard normal obeys $1-\Phi(x) \sim \frac{1}{x} \frac{1}{\sqrt{2\pi}}e^{-x^2/2}$
as $x \to \infty$.)

\textbf{Solution.}

We observe from the definition of $\limsup$, that for any $\omega$, we have that $\limsup \frac{|X_n(\omega)|}{\sqrt{\log n}} = \sqrt{2}$ if for any $\epsilon > 0$
\[
\omega \in \{ \frac{|X_n|}{\sqrt{\log n}} > \sqrt{2} - \epsilon \text{ i.o.} \} \setminus \{\frac{|X_n|}{\sqrt{\log n}} > \sqrt{2} + \epsilon \text{ i.o.}\} 
.\] 
Namely,
\[
\{\limsup \frac{|X_n|}{\sqrt{\log n}} = \sqrt{2}\} = \bigcap_{m = 1}^{\infty} \{ \frac{|X_n|}{\sqrt{\log n}} > \sqrt{2} - \frac{1}{m} \text{ i.o.} \} \setminus \{\frac{|X_n|}{\sqrt{\log n}} > \sqrt{2} + \frac{1}{m} \text{ i.o.}\} 
.\] 

For $c > 0$, and large $n$ so that $c \sqrt{ \log n}$ is large enough to use the given approximation for $1 - \Phi(x)$, we have
\begin{align*}
    P\left(\frac{|X_n|}{\sqrt{\log n}} > c\right) = P\left( |X_n| > c \sqrt{\log n} \right) &= 2(1 - \Phi \left( c \sqrt{\log n} \right)) \\
    &= \frac{2}{c \sqrt{\log n} \sqrt{2 \pi}} \exp \left( - c^2 \log n / 2 \right)  \\
    &= \frac{2}{c \sqrt{ \log n} \sqrt{2 \pi}} \left( \frac{1}{n} \right)^{c^2 / 2} 
.\end{align*}
By Rudin chapter 3 theorems 3.28 and 3.29, we have that $\sum_{n = 1}^{\infty} P\left( \frac{|X_n|}{\sqrt{\log n}} > c \right)$ diverges when $c \le \sqrt{2}$ and converges when $c > \sqrt{2}$.

Now, we apply the Borel Cantelli lemma.
Then,
\begin{align*}
    c > \sqrt{2} \implies \sum_{n = 1}^{\infty} P \left( \frac{|X_n|}{\sqrt{\log n}} > c \right) < \infty &\implies P\left( \frac{|X_n|}{\sqrt{\log n}} > c \text{ i.o.} \right) = 0 \\
    c \le \sqrt{2} \implies \sum_{n = 1}^{\infty} P \left( \frac{|X_n|}{\sqrt{\log n}} > c \right) = \infty &\implies P\left( \frac{|X_n|}{\sqrt{\log n}} > c \text{ i.o.} \right) = 1
.\end{align*}
By continuity from below, this implies
\[
P\left( \bigcap_{m = 1}^{\infty} \frac{|X_n|}{\sqrt{\log n}} > \sqrt{2} - \frac{1}{m} \text{ i.o.} \right)  = 1
.\] 
Thus,
\[
P(\limsup \frac{|X_n|}{\sqrt{\log n}} = \sqrt{2}) \ge  P\left( \bigcup_{m = 1}^{\infty} \frac{|X_n|}{\sqrt{\log n}} > \sqrt{2} - \frac{1}{m} \text{ i.o.}\right) - P \left(\bigcup_{m = 1}^{\infty} \frac{|X_n|}{\sqrt{\log n}} > \sqrt{2} + \frac{1}{m} \text{ i.o.} \right)  = 1 - 0
.\] 

\item
%Simon Analysis p.645
This problem shows that the conclusion of the second Borel--Cantelli Lemma holds under
the assumption that the events are pairwise independent only.  It is due to Erd\H{o}s and
R\'enyi, 1959.
\\
Suppose that the events $A_1,A_2,\ldots$ are pairwise independent, i.e., that
$P(A_i \cap A_j)= P(A_i)P(A_j)$.  Let $S_n=\sum_{j=1}^n \1_{A_j}$ and
$S_\infty =\sum_{j=1}^\infty \1_{A_j}$.
\begin{enumerate}
\item
Prove that ${\rm Cov}(\1_{A_i},\1_{A_j}) = P(A_i \cap A_j) - P(A_i)P(A_j)$.
\item
Prove that
\[
    {\rm Var}(S_n) = \sum_{j=1}^n (P(A_j) - P(A_j)^2) \le \sum_{j=1}^n P(A_j).
\]
\item
Prove (Chebyshev) that
\[
    P\big( S_n \le \frac 12 \sum_{j=1}^n P(A_j) \big) \le \frac{4}{\sum_{j=1}^n P(A_j)}.
\]
\item
Prove that
\[
    P\big( S_\infty \le \frac 12 \sum_{j=1}^n P(A_j) \big) \le \frac{4}{\sum_{j=1}^n P(A_j)}.
\]
\item
Prove that if $\sum_{j=1}^\infty P(A_j)=\infty$ then $P(S_\infty <\infty)=0$ and hence
$P(A_n\; {\rm i.o.})=1$.
\end{enumerate}

\textbf{Solution.}

\begin{enumerate}
    \item 
        We have $E(\1_{A_i}) = P(A_i)$. By linearity of expectation:
        \begin{align*}
            {\rm Cov} (\1_{A_i}, \1_{A_j}) &= E((\1_{A_i} - P(A_i))(\1_{A_j} - P(A_j)))\\
            &= E(\1_{A_i}\1_{A_j}) - P(A_i)P(A_j)
        .\end{align*}
        But then, $\1_{A_i} \1_{A_j} = \1_{A_i \cap A_j}$.
        So, ${\rm Cov}(\1_{A_i}, \1_{A_j}) = P(A_i \cap A_j) - P(A_i)P(A_j)$.
    \item We have:
        \begin{align*}
            {\rm Var}(S_n) = \sum_{j = 1}^{n} {\rm Var}(\1_{A_j}) + 2 \sum_{j < i} {\rm Cov}(\1_{A_i}, \1_{A_j})
                           &= \sum_{j = 1}^{n} E(\1_{A_j}^2) - E(\1_{A_j})^2\\
                           &= \sum_{j = 1}^{n} E(\1_{A_j}) - E(\1_{A_j})^2
                           = \sum_{j = 1}^{n} P(A_j) - P(A_j)^2
        .\end{align*}
    \item 
        \begin{align*}
            P\left( S_n \le  \frac{1}{2} \sum_{j = 1}^{n} P(A_j) \right) = P \left( \frac{1}{2} \sum_{j=1}^{n} P(A_j) \le \sum_{j=1}^{n} P(A_j) - S_n \right) 
            &= P \left( \frac{1}{2} \sum_{j=1}^{n} P(A_j) \le E(S_n) - S_n \right)  \\
            &\le  P \left( \frac{1}{2} \sum_{j=1}^{n} P(A_j) \le |S_n - E(S_n)| \right) 
        .\end{align*}
        Finally, we apply Chebyshev:
        \begin{align*}
            P \left( \frac{1}{2} \sum_{j=1}^{n} P(A_j) \le  |S_n - E(S_n)| \right) \le  \frac{4}{\left( \sum_{j=1}^{n} P(A_j) \right)^2 } {\rm Var}(S_n) = \frac{4}{\sum_{j=1}^{n} P(A_j)}
        .\end{align*}
    \item We have that $S_{\infty} = S_n + \sum_{j = n+1}^{\infty} \1_{A_j}$, since $\1_{A_j} \ge  0$.
        Therefore, $S_n \le  S_{\infty}$.
        However, this implies that for any $c \in \R$, $P(S_{\infty} \le c) \le  P(S_n \le  c)$, since if $\omega \in \{S_{\infty} \le  c\} $ in other words $S_{\infty}(\omega) \le c$, then $S_n(\omega) \le  S_{\infty} (\omega) \le  c$ so that $\omega \in \{S_n \le  c\} $.
        Therefore, 
        \[
        P\left( S_{\infty} \le  \frac{1}{2} \sum_{j=1}^{n} P(A_j) \right) \le  P\left( S_n \le  \frac{1}{2} \sum_{j=1}^{n} P(A_j) \right) \le \frac{4}{\sum_{j=1}^{n} P(A_j)}
        .\] 
    \item We show that $P(S_{\infty} \le  N) = 0$ for all $N \in \N$.
        Since $\sum_{j=1}^{\infty} P(A_j) = \infty$, there exists some $n$ such that $ \frac{1}{2}\sum_{j = 1}^{n} P(A_j) \ge N$.
        Likewise, for any $\epsilon > 0$, there exists some $n'$ such that $4 / \sum_{j=1}^{n'} P(A_j) < \epsilon$. Let $n'' = \max \{ n, n'\}$.
        But then,
        \[
        P(S_{\infty} < N) \le P(S_{\infty} \le  \frac{1}{2}\sum_{j=1}^{n''}P(A_j)) \le \frac{4}{\sum_{j=1}^{n''} P(A_j)} < \epsilon
        .\] 
        Since $\epsilon$ was arbitrary, $P(S_{\infty} < N) = 0$.
        By continuity from below, we have $P(S_{\infty} < \infty) = 0$.
\end{enumerate}

\item
Let $X \ge 0$ be a nonnegative random variable with $EX^2<\infty$.
\begin{enumerate}
\item
%Durrett p.15 #3.8
Prove that
\[
    P(X>0) \ge \frac{(EX)^2}{EX^2}.
\]
Hint: Consider the random variable $X\1_{X>0}$.
\item
%Paley-Zygmund inequality
Prove that, for $\theta \in [0,1]$,
\[
    P(X > \theta EX) \ge (1-\theta)^2 \frac{(EX)^2}{EX^2}.
\]
\end{enumerate}

\textbf{Solution.}
\begin{enumerate}
    \item We consider the hint. In particular, $X \1_{X > 0} = X$. Applying the Cauchy Schwartz inequality,
        \[
        E(X) = E(X \1_{X > 0}) \le \sqrt{E(X^2)E(\1_{X > 0}^2)} = \sqrt{E(X^2) P(\1_{X > 0})}
        .\] 
        In conclusion, $P(X > 0) \ge  E(X)^2 / E(X^2)$.
    \item We apply a very similar application of the C.S. inequality.
        In particular, note that $E[X - E[\theta X]] \le  E (X - E[\theta X])^{+} = E X \1_{X \ge  \theta X}$.
        Then:
        \begin{align*}
            (1 - \theta)E[X] = E[X - E[\theta X]] \le E X \1_{X \ge \theta E[X]} \le \sqrt{EX^2 P(X \geq  \theta E[X])}
        .\end{align*}
        Rearranging, we get: $P(X \geq \theta E[X]) \ge (1 - \theta)^2 E[X]^2 / EX^2$.
\end{enumerate}

\item
%Durrett Theorem 7.2
Let $X_1,X_2,\ldots$ be i.i.d.\ with $EX_i^+=\infty$ and $EX_i^-<\infty$.
Let $S_n=X_1+\cdots +X_n$.  Prove that $n^{-1}S_n \to \infty$ a.s.
\\
Hint: Apply SLLN to $X_{i,N}=\min\{X_i,N\}$.

\textbf{Solution.}

First, we see that $E X_{i, N}^{-} = E X_{i}^{-} < \infty$, and $E X_{i, N}^{+} \le  E(N) \le  N$. Therefore, $E X_{i, N} < \infty$. Since $X_{i, N}$ are i.i.d., we can apply the SLLN to $X_{i, N}$. Let $\mu_N = E X_{i, N}$.
Now, we have that $X_{i, N} \to X_{i}$ pointwise: indeed, for any $\omega \in \Omega$, we have that $X_{i}(\omega) < M$ for some $M \in \N$, and then $X_{i, N}(\omega)  = X_{i}(\omega)$ for all $N > M$.
By the Monotone Convergence Theorem, we have that: $E X_{i, N}^{+} \to E X_i^{+}$ as $N \to \infty$.
In particular, $\mu_N = E X_{i, N}^{+} - E X_{i, N}^{-} \to \infty$ as $N \to \infty$.

Now, pointwise we have $X_i \ge  \min \{X_i, N\} $.
Therefore, at each point, 
\[
S_n = \sum_{j = 1}^{n} X_j \ge  \sum_{j = 1}^{n} X_{j, N}
.\] 
In particular, $P(\lim_{n \to \infty} n^{-1}(\sum_{j=1}^{n} X_{j, N})  = \mu_N) = 1$ implies that $P(\lim_{n \to \infty} n^{-1}S_n \ge \mu_N) = 1$.
But then, by continuity from above, since these sets are decreasing, and since $\mu_N \to \infty$ as $N \to \infty$, we have $P(\lim_{n \to \infty} n^{-1} S_n = \infty)$ or $n^{-1}S_n \to \infty$ a.s.

\item
Let $X$ and $Y$ be random variables with finite mean. Suppose that for any $A\in \Fcal$ we have $E[X\1_A] \le E[Y\1_A]$.  Prove that $X\le Y$ a.s.

\textbf{Solution.}

Suppose for the sake of contradiction that $P(X - Y > 0) > 0$.
Then, there exists some $n$ such that $P\left(X - Y > \frac{1}{n}\right) > 0$, since otherwise we would have $P(X - Y > 0) = 0$ by continuity of probabilities.
Then, there exists $\epsilon > 0$ such that $P\left( X - Y > \frac{1}{n} \right) \ge  \epsilon$.
Let $A = \{X - Y > \frac{1}{n}\} $.
But then 
\[
E[X \1_{A}] - E[Y \1_{A}] = E[X \1_{A}- Y \1_{A}] \ge \frac{1}{n} E[\1_{A}] = \frac{1}{n} P(A) \ge \frac{\epsilon}{n} > 0
.\] 
where in the first step we used the fact that $X - Y > \frac{1}{n}$ and the order preserving property of expectation.
However, this contradicts that $E[X \1_{A}] \le  E[Y \1_{A}]$, since this i the case for any $A \in \mathcal{F}$.
Therefore, it must be the case that $P(X - Y > 0) = 0$, in other words, $P(X \le  Y)$ a.s.

\end{enumerate}

 

\bigskip  \noindent
{\bf Recommended problems.}
The following problems from Rosenthal are recommended but are not to be handed in:

4.5.4, 4.5.7, 6.3.2, 6.3.4, 6.3.6., 5.5.6, 5.5.9.
\\
For solutions to even-numbered problems see: \url{http://www.probability.ca/jeff/grprobbook.html}.

\smallskip\noindent
This problem (do not hand in)
gives a probabilistic proof of Weierstrass's Theorem, due to Bernstein (1912):
Let $f:[0,1]\to\R$ be a continuous function.
For $x \in [0,1]$, let $S_n$ have a Bin($n,x$) distribution and set $X_n=n^{-1}S_n$.
Prove that $q_n(x) = E f(X_n)$ is a polynomial in $x$
and that $q_n$ converges uniformly to $f$ on $[0,1]$.
\\
Hint: Write $f(x)-q_n(x)$ as a sum over $m \in \{0,\ldots,n\}$, divide the sum over
$m$ into $|\frac mn -x| \le \delta$ and $|\frac mn -x|>\delta$, and use the Chebyshev
inequality to bound the latter part of the sum.



\bigskip\noindent
{\bf Two simulations.}  (There is nothing for you to produce concerning this.)
The standard normal with density $f(x) = \frac{1}{\sqrt{2\pi}}e^{-x^2/2}$ has mean zero.
The Cauchy distribution with density $f(x) = \frac{1}{\pi}\frac{1}{1+x^2}$ has undefined mean
(this distribution arose in Assignment~4 \#1).

\includegraphics[scale=0.32]{418-5-24_stdnormal.png}
\includegraphics[scale=0.32]{418-5-24_Cauchy.png}




\bigskip \noindent
Quote of the week:
\emph{It is the mark of a truly intelligent person to be moved by statistics.}

\hspace*{\fill}
George Bernard Shaw



\end{document}
