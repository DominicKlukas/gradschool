\documentclass[hidelinks]{article}[12pt]

\oddsidemargin 3mm
\evensidemargin 3mm
\topmargin -12mm
\textheight 660pt
\textwidth 450pt


\usepackage{hyperref}
\usepackage{url}
\usepackage{amstext}
\usepackage{amsmath,amssymb}
\usepackage{graphicx}
\usepackage{bbm}

% Some basic packages
\usepackage[utf8]{inputenc}
\usepackage[T1]{fontenc}
\usepackage{textcomp}
\usepackage{url}
\usepackage{graphicx}
\usepackage{float}
\usepackage{booktabs}
\usepackage{enumitem}
\usepackage{bbm}

\pdfminorversion=7

% Don't indent paragraphs, leave some space between them
\usepackage{parskip}

% Hide page number when page is empty
\usepackage{emptypage}
\usepackage{subcaption}
\usepackage{multicol}
\usepackage{xcolor}

% Other font I sometimes use.
% \usepackage{cmbright}

% Math stuff
\usepackage{amsmath, amsfonts, mathtools, amsthm, amssymb}
% Fancy script capitals
\usepackage{mathrsfs}
\usepackage{cancel}
% Bold math
\usepackage{bm}
% Some shortcuts
\newcommand\N{\ensuremath{\mathbb{N}}}
\newcommand\R{\ensuremath{\mathbb{R}}}
\newcommand\Z{\ensuremath{\mathbb{Z}}}
\renewcommand\O{\ensuremath{\emptyset}}
\newcommand\Q{\ensuremath{\mathbb{Q}}}
\newcommand\C{\ensuremath{\mathbb{C}}}

\DeclareMathOperator{\var}{Var}
\DeclareMathOperator{\ani}{Ann}
\DeclareMathOperator{\mnn}{Min}
\DeclareMathOperator{\spn}{span}
\DeclareMathOperator{\ch}{ch}
\DeclareMathOperator{\aut}{Aut}
\DeclareMathOperator{\ide}{Id}
\DeclareMathOperator{\iso}{Iso}
\DeclareMathOperator{\sinc}{sinc}


% Easily typeset systems of equations (French package)
\usepackage{systeme}

% Put x \to \infty below \lim
\let\svlim\lim\def\lim{\svlim\limits}

%Make implies and impliedby shorter
\let\implies\Rightarrow
\let\impliedby\Leftarrow
\let\iff\Leftrightarrow
\let\epsilon\varepsilon

% Add \contra symbol to denote contradiction
\usepackage{stmaryrd} % for \lightning
\newcommand\contra{\scalebox{1.5}{$\lightning$}}

% \let\phi\varphi

% Command for short corrections
% Usage: 1+1=\correct{3}{2}

\definecolor{correct}{HTML}{009900}
\newcommand\correct[2]{\ensuremath{\:}{\color{red}{#1}}\ensuremath{\to }{\color{correct}{#2}}\ensuremath{\:}}
\newcommand\green[1]{{\color{correct}{#1}}}

% horizontal rule
\newcommand\hr{
    \noindent\rule[0.5ex]{\linewidth}{0.5pt}
}

% hide parts
\newcommand\hide[1]{}

% si unitx
\usepackage{siunitx}
\sisetup{locale = US}

% Environments
\makeatother
% For box around Definition, Theorem, \ldots
\usepackage{mdframed}
\mdfsetup{skipabove=1em,skipbelow=0em}
\theoremstyle{definition}
\newmdtheoremenv[nobreak=true]{definition}{Definition}
\newmdtheoremenv[nobreak=true]{property}{Property}
\newmdtheoremenv[nobreak=true]{consequence}{Consequence}
\newmdtheoremenv[nobreak=true]{lemma}{Lemma}
\newmdtheoremenv[nobreak=true]{proposition}{Proposition}
\newmdtheoremenv[nobreak=true]{theorem}{Theorem}
\newmdtheoremenv[nobreak=true]{law}{Law}
\newmdtheoremenv[nobreak=true]{corollary}{Corollary}
\newmdtheoremenv{conclusion}{Conclusion}
\newmdtheoremenv{bonus}{Bonus}
\newmdtheoremenv{conjecture}{Conjecture}
\newtheorem*{recall}{Recall}
\newtheorem*{intermezzo}{Intermezzo}
\newtheorem*{notation}{Notation}
\newtheorem*{observation}{Observation}
\newtheorem*{exercise}{Exercise}
\newtheorem*{comment}{Comment}
\newtheorem*{practice}{practice}
\newtheorem*{problem}{Problem}
\newtheorem*{terminology}{Terminology}
\newtheorem*{application}{Application}
\newtheorem*{eg}{Example}
\newtheorem*{question}{Question}
\newtheorem*{previouslyseen}{As previously seen}
\newtheorem*{remark}{Remark}
\newtheorem*{observe}{Observe}
\newtheorem*{intuition}{Intuition}
\newtheorem*{solution}{Solution}

% End example and intermezzo environments with a small diamond (just like proof
% environments end with a small square)
\usepackage{etoolbox}
\AtEndEnvironment{vb}{\null\hfill$\diamond$}%
\AtEndEnvironment{intermezzo}{\null\hfill$\diamond$}%
% \AtEndEnvironment{opmerking}{\null\hfill$\diamond$}%

% Fix some spacing
% http://tex.stackexchange.com/questions/22119/how-can-i-change-the-spacing-before-theorems-with-amsthm
\makeatletter
\def\thm@space@setup{%
  \thm@preskip=\parskip \thm@postskip=0pt
}


% Exercise 
% Usage:
% \oefening{5}
% \suboefening{1}
% \suboefening{2}
% \suboefening{3}
% gives
% Oefening 5
%   Oefening 5.1
%   Oefening 5.2
%   Oefening 5.3
%\newcommand{\exc}[1]{%
    %\def\@excercise{#1}%
    %\subsection*{Excercise #1}
%}

%\newcommand{\sub-excercise}[1]{%
    %\subsubsection*{Excercise \@excercise.#1}
%}


% \lecture starts a new lecture (les in dutch)
%
% Usage:
% \lecture{1}{di 12 feb 2019 16:00}{Inleiding}
%
% This adds a section heading with the number / title of the lecture and a
% margin paragraph with the date.

% I use \dateparts here to hide the year (2019). This way, I can easily parse
% the date of each lecture unambiguously while still having a human-friendly
% short format printed to the pdf.

%\usepackage{xifthen}
%\def\testdateparts#1{\dateparts#1\relax}
%\def\dateparts#1 #2 #3 #4 #5\relax{
    %\marginpar{\small\textsf{\mbox{#1 #2 #3}}}
%}

%\def\@lecture{}%
\newcommand{\note}[3]{
    \ifthenelse{\isempty{#3}}{%
        \def\@note{#1}%
    }{%
        \def\@note{#1: #3}%
    }%
    \subsection*{\@note}
    \marginpar{\small\textsf{\mbox{#2}}}
}



% These are the fancy headers

\makeatother


% Todonotes and inline notes in fancy boxes
\usepackage{tcolorbox}

% Make boxes breakable
\tcbuselibrary{breakable}

% Verbetering is correction in Dutch
% Usage: 
% \begin{verbetering}
%     Lorem ipsum dolor sit amet, consetetur sadipscing elitr, sed diam nonumy eirmod
%     tempor invidunt ut labore et dolore magna aliquyam erat, sed diam voluptua. At
%     vero eos et accusam et justo duo dolores et ea rebum. Stet clita kasd gubergren,
%     no sea takimata sanctus est Lorem ipsum dolor sit amet.
% \end{verbetering}
\newenvironment{correction}{\begin{tcolorbox}[
    arc=0mm,
    colback=white,
    colframe=green!60!black,
    title=Remark,
    fonttitle=\sffamily,
    breakable
]}{\end{tcolorbox}}



% Figure support as explained in my blog post.
\usepackage{import}
\usepackage{xifthen}
\usepackage{pdfpages}
\usepackage{transparent}
\newcommand{\incfig}[1]{%
    \def\svgwidth{\columnwidth}
    \import{./figures/}{#1.pdf_tex}
}

% Fix some stuff
% %http://tex.stackexchange.com/questions/76273/multiple-pdfs-with-page-group-included-in-a-single-page-warning
\pdfsuppresswarningpagegroup=1

% My name
\author{Dominic Klukas}


\newcommand{\Fcal}{{\cal F}}
\newcommand{\E}{\mathbb{E}}
\newcommand{\1}{\mathbbm{1}}
\newcommand{\deq}{\stackrel{d}{=}}
\newcommand{\dto}{\stackrel{d}{\to}}
\newcommand{\pto}{\stackrel{\mathrm{p}}{\to}}
\newcommand{\asto}{\stackrel{\mathrm{a.s.}}{\to}}

\begin{document}


%\pagestyle{empty}

\noindent
{\bf Math 418/544   Assignment 8} \hfill  November 25, 2024

\bigskip \noindent
Dr.\ J.\ Hermon


Dominic Klukas
Student Number: 64348378

\bigskip \noindent
{\bf This assignment is due in Canvas at 23:59 on Friday, Dec 5th. \\
\emph{Late assignments are not accepted.}
\\
Submit a solution to 5 out of the 10 questions.}

We solve problems 1, 2, 3, 8, 10

\begin{enumerate}
	
	\item  Let $g:[0,\infty) \to [0,\infty)$ be such that $\lim_{x \to \infty}\frac{g(x)}{x}=\infty$. Let $(X_j:j \in J)$ be a collection of random variables satisfying that $\sup_{j \in J}\mathbb{E}[g(|X_j|)]<\infty$. Show that  $(X_j:j \in J)$ is uniformly integrable.
	
	\textbf{Hint:} $|x| \1_{|x|>M} \le g(|x|)\delta_g(M)$ for all $M>0$ for some appropriate $\delta_g:(0,\infty) \to (0,\infty)$.


    \textbf{Solution.}
    Let $\epsilon > 0$.
    Then, $C = \sup_{j \in J} \mathbb{E}[g(|X_j|)]$, so $C < \infty$.
    Let $L = C / \epsilon$.
    Since $\lim_{x \to \infty} \frac{g(x)}{x} = \infty$, we have that there exists some $M \in \R$ such that for all $x > M$, $g(x) / x > L$.
    However, this implies
    \[
        |x| \1_{|x| > M} < \frac{1}{L} g(|x|) \1_{|x| > M} \leq \frac{1}{L} g(|x|)
    ,\]
    where the last inequality follows because $g \geq 0$.
    Then, for all $j \in J$, we have
    \[
        \mathbb{E}[|X_j| \1_{|X_j| > M}] < \frac{1}{L} \mathbb{E}[g(|X_j|)] \leq \frac{C}{L} < \epsilon
    ,\]
    so that $\sup_{j \in J} \mathbb{E}[|X_j| \1_{|X_j| > M}] < \epsilon$.
    Since $\epsilon > 0$ was arbitrary, it follows that 
    \[
    \lim_{M \to \infty} \sup_{j \in J} \mathbb{E}[|X_j| \1_{|x| > M}] < \epsilon
    ,\] so that $(X_j : j \in J)$ is uniformly integrable.


\item For $n \in \mathbb{N}$ let $\sigma_n$ be a random permutation of $[n]:=\{1,2,\ldots,n\}$ picked uniformly at random (i.e., its distribution is the uniform distribution on the set of all $n!$ permutation of $[n]$). Let $X_n:=|\{j \in [n]:\sigma_n(j)=j\}|$ be the number of fixed points of $\sigma_n$. Use the method of moments to show that $X_n \Rightarrow \mathrm{Pois}(1)$ (justify why it applies; you may look up online the MGF of the Poisson distribution). 
\\ You may rely without a proof on the fact that if $Y_n \sim \mathrm{Bin}(n,\frac{1}{n})$ and $Y \sim \mathrm{Pois}(1)$ then $\lim_{n \to \infty}\mathbb{E}[Y_n^k]=\mathbb{E}[Y^k]$ for all $k \in \mathbb{Z}_+$.


\textbf{Hint:} Use indicator decompositions of $X_n$ and of $Y_n$. Expand $X_n^k$ and of $Y_n^k$ to a sum of monomials. Use the fact that $x^{\ell}=x$ for $x \in \{0,1\}$ and $\ell \ge 1$.

\textbf{Solution.}

We compute the moments of $X_n$ and $Y_n$, and then show that they both converge to the same value.
As we are given that the moments of $Y_n$ converge to the moments of $Y$, it follows that the moments of $X_n$ converge to the moments of $Y$. 


Consider the $k$-th moment.
First, we take the hint.
Notice that $X_n = \sum_{j = 1}^n I_j$, where $I_j = \1_{\{\sigma_n(j) = j\}}$.
Then, for $X_n$, by linearity of expectation we have:
\begin{align*}
    \mathbb{E}[X_n^k] &= \mathbb{E}\left[ \left( \sum_{j=1}^{n} I_j \right)^k \right]  \\
                      &= \sum_{\substack{(j_1, \ldots, j_k)\\ j_i \in \{1, \ldots, n\} }} \mathbb{E}\left[ \prod_{i=1}^{k} I_{j_i}  \right] \label{eq:sum} \\
\end{align*}
Here, we have every combination of $j_1, \ldots, j_k$ such that $j_i \in \{1, \ldots, n\} $.
Now, since $I_j\cdot I_j = I_j$, we have that for any sequence containing distince $j_1, \ldots, j_r$, we will then have
\[
I_{j_1}^{n_1} \cdot \ldots \cdot I_{j_r}^{n_r} = I_{j_1} \cdot \ldots \cdot I_{j_r}, \sum_{i = 1}^{r} n_i = k
.\] 
Also, we compute:
\[
    \mathbb{E}[I_{j_1} \cdot \ldots \cdot I_{j_r}] = \frac{(n-r)!}{n!}
,\]
since if we have $r$ elements fixed, there are $(n-r)!$ permutations that permute the remaining elements.
If we fix some distinct set $j_1, \ldots, j_r$, we see that there is some unique number $n(r, k)$, the number of terms which are a product of $k$ of these $I_j$, if the product contains $r$ distinct factors $I_{j_1}, \ldots, I_{j_r}$.
Next, if there are $n$ total terms in the sum $X_n$, then we can see that there will be $n(n-1)\cdot\ldots \cdot (n-r+1) = \frac{n!}{(n-r)!}$ different choices for $r$ distinct factors $I_j$.
Then, we group the terms in the sum \eqref{eq:sum} based on how many distinct 
\begin{align*}
    \sum_{\substack{(j_1, \ldots, j_k)\\ j_i \in \{1, \ldots, n\} }} \mathbb{E}\left[ \prod_{i=1}^{k} I_{j_i}  \right] &= \sum_{r = 1}^k n(r,k) \sum_{j_1 < \ldots < j_r} \mathbb{E}[I_{j_1} \cdot \ldots \cdot I_{j_r}]\\
                                                                                                                       &= \sum_{r=1}^k n(r, k) \sum_{j_1 < \ldots < j_r} \frac{(n-r)!}{n!}\\
                                                                                                                       &= \sum_{r=1}^k n(r, k) \frac{n!}{(n-r)!} \frac{(n-r)!}{n!} = \sum_{r=1}^{k} n(r, k)
.\end{align*}
Notice, that this is independent of $n$, so we have that $\lim_{n \to \infty} \mathbb{E}[X_n^k] = \sum_{r=1}^{k}n(r, k)$.

Now, we do a similar computation for the Binomial random variable $B(n, 1 / n)$ random variable.
We have that $B(n, 1 / n) = \sum_{j = 1}^n B_j$, where $B_j(1 / n)$ is the bernoilli random variable with probability $1 / n$ of success.
Then, we have $B_j(1 / n)^n_j = B_j(1 / n)$, and so for $j_1, \ldots, j_r$ distinct, we have by independence of these bernoilli trials, that
\[
    \mathbb{E}[B_{j_1}(1 / n)^{n_{j_1}} \cdot \ldots \cdot B_{j_r}(1 / n)^{n_{j_r}} ] = \mathbb{E}[B_{j_1}] \cdot \ldots \cdot \mathbb{E}[B_{j_r}(1 / n)] = (1 / n)^{r}
.\] 
Then, a similar computation as for the $X_n$ moment calculation gives us

\[
\mathbb{E}[Y_{n}^{k}] = \sum_{r=1}^{k} n(r, k) \sum_{j_1 < \ldots < j_r} \left( \frac{1}{n} \right)^{r} = \sum_{r=1}^{k} n(r, k) \frac{n!}{(n-r)!} \frac{1}{n^r}
.\] 
But then as $n \to \infty$, we have that $\frac{n!}{(n-r)!} \to n^r$, so $E[Y_n^k] \to \sum_{r=1}^{k} n(r, k) = E[X_n^k]$.

Now, since the MGF of the poisson (1) distribution is $\exp(e^t - 1)$, we have that for any $s_0 > 0$, $\exp(e^s - 1)$ is finite for all $|s| < s_0$.
Then, by theorem 11.4.3, the Poisson distribution is determined by its moments.
Therefore, if a sequence of random variables is such that its moments converge to the moments of the Poisson distribution, then the random variables converge to the Poisson distribution in distribution.
Since we are given that $\mathbb{E}[Y_n^k] \to \mathbb{E}[\text{Pois}(1)^k]$, and $\mathbb{E}[X_n^k] \to \lim_{n \to \infty} \mathbb{E}[Y_n^{k}]$, we have that $\mathbb{E}[X_n^{k}] \to \mathbb{E}[\text{Pois}(1)^{k}]$, so $X_n$ converges to $\text{Pois}(1)$ in distribution.

\item
\begin{enumerate}
\item
Suppose that the sequence of probability measures $(\mu_n)_{n=1}^{\infty}$ is tight.
Show that their characteristic functions are uniformly equicontinuous, i.e., for all $\varepsilon >0$
there exists $\delta >0$ such that if $|h|<\delta$
then for all $n$ and all $t \in \mathbb{R}$ 
\[|\phi_n(t+h)-\phi_n(t)| < \infty \, ,\]
where $\phi_n$ is the  characteristic function of $\mu_n$.

\textbf{Hint:} You may rely on the inequality $|1-e^{-iy}| \le \min\{2,|y|\}$ for all $y \in \mathbb{R}$. Deduce from it that for all $x,h \in \mathbb{R}$,  $|1-e^{-ihx}| \le |h|^{1/2}+2\1_{|x|>|h|^{-1/2}}  $.
\item
Suppose that $\mu_n \Rightarrow \mu$. Denote the  characteristic function of $\mu$ by $\phi$ and of $\mu_n$ by $\phi_n$.  Use (a) to conclude that $\phi_n(t) \to \phi(t)$
uniformly on compact sets, i.e., for all $M>0$ and $\varepsilon >0$ there exists $N=N(\varepsilon,M)$ such that for all $n \ge N$ we have that $|\phi_n(t) - \phi(t)|<\varepsilon$ for all $t \in [-M,M]$.
\item
Give an example to show that the convergence in (b) need not be uniform on the entire
real line.
\end{enumerate}

\textbf{Solution.}

\begin{enumerate}
    \item We compute:
        \begin{align*}
            \left| \phi_{n}(t + h) - \phi_{n}(t) \right|  &= \left| \int \exp(i(t + h)x) - \exp(it x) \mu_n(dx) \right| \\
            &\leq \int \left| \exp(i(t + h)x) (1 - \exp(-ihx)) \right| \mu_n(dx) = \int \left| 1 - \exp(-ihx) \right| \mu_n (dx)
        .\end{align*}
        Now, we are given that $|1 - e^{ihx}| \leq \min \{2, |hx|\} $.
        Then, consider that for $|x| < \frac{1}{|h|^{1 / 2}}$, we have that $|h x| < |h|^{ 1 /2}$.
        But then, we have:
        \[
        \min \{|hx|, 2\}  \leq |h|^{1 / 2} \1_{|x| < \frac{1}{|h|^{1 / 2}}} + \1_{|x| \ge \frac{1}{|h|^{1 / 2}}}  \leq |h|^{ 1 / 2} + \1_{|x| \geq \frac{1}{|h|^{1 / 2}}}
        .\] 
        We can then split this integral apart:
        \begin{align*}
            \int |1 - \exp(-ihx)| \mu_n (dx) &\leq \int |h|^{1 / 2} \mu_n (dx) + \int \1_{|x| \geq \frac{1}{|h|^{1 / 2}}}  \mu_n(dx)
        .\end{align*}
        Let $\epsilon > 0$.
        Now, since $(\mu_n)_{n=1}^{\infty}$ is tight, we have that there exists some $R$ such that $\int \1_{|x| \geq R} \mu_n(dx) < \epsilon / 2$.
        Then, let $\delta < \min\{ (\epsilon / 2)^2, \frac{1}{R^2} \}$.
        For all $|h| < \delta$, we then have
        \[
        \int |h|^{1 / 2} \mu_n (dx) + \int \1_{|x| \geq \frac{1}{|h|^{1 / 2}}}  \mu_n(dx) < \epsilon / 2 + \int \1_{|x| \geq R} \mu_n(dx)  \leq \epsilon
        .\]
        Therefore, the characteristic functions are uniformly euquicontinuous, since $t, \epsilon$ and we have produced $\delta$ as required.
    \item Let $\epsilon > 0$.
        Since equicontinuous, there exists some $\delta$ such that for all $t \in [-M, M]$, and all $n$, $|\phi_n(t + h) - \phi_n(t)| < \epsilon / 3$ for all $|h| < \delta$.

        Since $[-M, M]$ is compact, and $\phi(t)$ is continuous,we have that $\phi(t)$ is uniformly continuous on $[-M, M]$ (this is a basic result from Math 321).
        Therefore, there exists some $\delta'$ such that $|\phi(t + h) - \phi(t)| < \epsilon / 3$ for all $|h| < \delta$.

        Now, since $\phi_n(t) \to \phi(t)$ pointwise, we have that each each $t$ there exists some $L$ such that for all $n > L$, $|\phi_n(t) - \phi(t) | < \epsilon / 3$.
        Call this number $L(t)$.
        Let $\delta'' = \min \{\delta, \delta'\} $.
        Then, consider the set of open sets $\{N(t,\delta'' )\}, t \in [-M, M] $.
        This is clearly an open cover of $[-M, M]$, since it contains the balls centered at each $t \in [-M, M]$.
        Since $[-M, M]$ is compact, there is a finite open subcover $\{N(t_1, \delta''), \ldots, N(t_s, \delta'')\}$.
        Let $L = \max \{ L(t_1), \ldots, L(t_s)\}$.
        In particular, for all $n > L$, $|\phi_n(t_i) - \phi(t_i)| < \epsilon / 3$ for any $1 \leq i \leq s$.

        Putting it all together, let $t \in [-M, M]$. Let $n > L$.
        Then, there exists some $1 \leq i \leq s$ such that $t \in N(t_{i}, \delta'')$.

        Since $|t_i - t| < \delta$, uniform equicontinuity of $\phi_n$ means that we have
        \[
        |\phi_{n}(t_i) - \phi_n(t)| < \epsilon / 3
        .\] 
        Since $ n > L$, we have 
        \[
        |\phi_n(t_i) - \phi(t_i)| < \epsilon / 3
        .\] 
        Finally, since $|t_i - t| < \delta'$, we have
        \[
        |\phi(t) - \phi(t_i)| < \epsilon / 3
        .\] 
        By triangle inequality,
        \[
        |\phi_n(t) - \phi(t)| = |\phi_{n}(t) - \phi_n(t_i) + \phi_n(t_i) - \phi(t_i) + \phi(t_i) - \phi(t) | \leq \epsilon / 3 + \epsilon / 3 + \epsilon / 3 = \epsilon
        .\] 
        Therefore, $\phi_n \to \phi$ uniformly on compact sets.
    \item For an example where converge in (b) need not be uniform on the entire real line, we need to find a distributtion $\mu_n \implies \mu$ such that $(\mu_n)_{n=1}^{\infty}$ is tight, but $\phi_n \to \phi$ doesn't converge uniformly on the entire real line.

        Let $\mu_n$ be the normal distribution, with mean $0$ and standard deviation $\frac{1}{n}$.
        First, we show that these distributions are tight.
        For $\mu_1$, we have that the CDF is given by $\Phi(x)$, since $\mu_1$ is the standard normal distribution.
        Let $\epsilon > 0$.
        Since $\lim_{x \to - \infty}\Phi(x) = 0$, there must be some $R$ such that $\Phi(-R) < \epsilon / 2$, and so $1 - \Phi(R) < \epsilon / 2$.
        But then, for all $n > 1$, the CDF of $\mu_n$ is given by $\Phi( x / (1 / n)) = \Phi(n x)$.
        But then, we have that $\Phi(n (-R)) < \Phi(-R) < \epsilon / 2$, and similarly, $1 - \Phi(nR) < \Phi(R) < \epsilon / 2$.

        Therefore, for all $n$, we have that there exists $R$ such that $\int_{|x| > R} \mu_n(dx) < \epsilon$, so $(\mu_n)$ is tight.

        Next, we observe that the probability desnity function of $\mu_n$ is given by $f_n(x) = \frac{n}{\sqrt{2\pi}} \exp(-n^2x^2 / 2)$.
        We computed the characterisitc function of the normal distribution in class: $\phi_n(t) = \exp(- t^2 / 2n^2)$
        We can see that, as $n \to \infty$, $\phi_n(t) = 1$.
        However, this clearly does not converge uniformly, since for any $n$, by taking $t \to \infty$, we can find some $t'$ sufficiently large so that $|1 - \exp(-(t')^2 / 2n^2)| > \frac{1}{2}$.
        Therefore, $\phi_n(t) \to 1$, but not uniformly.
\end{enumerate}


\item \textbf{Parts (i)--(v) of this question are unmarked} (you can skip them and rely on them):
\\ Let $\lambda>0$. Consider for all $n$ a sequence $(p_{n,m}: 1 \le m \le N_n )$ of real numbers in $[0,1]$ satisfying that (1) $\sum_{m=1}^{N_n}p_{n,m}=\lambda$ and (2) $\lim_{n \to \infty}p_*(n)=0$, where $p_*(n):=\max_{ 1 \le m \le N_n}p_{n,m}$. 



(i) Show that $\lim_{n \to \infty} \sum_{m=1}^{N_n}p_{n,m}^2 \le \lim_{n \to \infty} p_*(n) \sum_{m=1}^{N_n}p_{n,m}=0$.  \\ 

(ii) Suppose that $(q_{n,m}: 1 \le m \le N_n) $ satisfy that  for some $C>0$  we have that $|q_{n,m}-p_{n,m}| \le Cp_{n,m}^2$ for all $n \ge 1$ and all $m \in [N_n]:=\{1,2,\ldots,N_n\}$. Show that  $\lim_{n \to \infty}\sum_{m=1}^{N_n}q_{n,m}=\lambda$ and that $\lim_{n \to \infty}q_*(n)=0$, where $q_*(n):=\max_{ 1 \le m \le N_n}q_{n,m}$.

(iii)  Let $q_{n,m}$ be the solution to  $\mathbb{P}[\mathrm{Pois}(q_{n,m})=1]=q_{n,m}e^{-q_{n,m}}=p_{n,m}$.\footnote{Since $p_*(n) \to 0$, w.l.o.g.\ we may assume that $p_*(n) \le e^{-1}$ for all $n$, and this ensures that such a solution $q_{n,m}$ exists and that it lies in $(0,1]$.} Show that for some $C>0$ we have for all $n \ge 1$ and all $m \in [N_n]$ that $|q_{n,m}-p_{n,m}| \le Cp_{n,m}^2$ and $\mathbb{P}[\mathrm{Pois}(q_{n,m}) \ge 2]=1-e^{-q_{n,m}}-q_{n,m}e^{-q_{n,m}}  \le Cp_{n,m}^2$.

(iv) Suppose that $\lambda_n>0$ and $\lim_{n \to \infty} \lambda_n = \lambda$. Show that $\mathrm{Pois}(\lambda_n) \Rightarrow \mathrm{Pois}(\lambda)$.  

(v) Let $q_{n,m}$ be as in (iii). For any fixed $n \ge 1$, let $(Y_{n,m}: m \in [N_n])$ be independent Poisson random variables, where $Y_{n,m} \sim \mathrm{Pois}(q_{n,m})$ for all $n \ge 1$ and all $m \in [N_n]$. Show that $Y(n):=\sum_{m=1}^{N_n}Y_{n,m} \Rightarrow \mathrm{Pois}(\lambda)$. 

(vi) Show that \[\lim_{n \to \infty}\mathbb{P}\left[\sum_{m=1}^{N_n}Y_{n,m}\1_{Y_{n,m}  \ge 2} \neq 0 \right]=\lim_{n \to \infty}\mathbb{P}[Y_{n,m} \ge 2 \text{ for some }m \in [N_n]]=0 \, .\] Deduce that $Z(n) :=- \sum_{m=1}^{N_n}Y_{n,m}\1_{Y_{n,m} > 1}=-\sum_{m=1}^{N_n}Y_{n,m}\1_{Y_{n,m} \ge 2} \to 0$ in distribution, as $n \to \infty$. Use this and part (vi) to deduce that \[Y(n)+Z(n)=\sum_{m=1}^{N_n}Y_{n,m}\1_{Y_{n,m} \le  1}=\sum_{m=1}^{N_n}\1_{Y_{n,m} =  1} \Rightarrow \mathrm{Pois}(\lambda) \, .\] 


(vii) For all (fixed) $n \ge 1$ let $(X_{n,m}: 1 \le m \le N_n)$ be independent Bernoulli random variables with $\mathbb{P}[X_{n,m}=1]=p_{n,m}=1-\mathbb{P}[X_{n,m}=0]$ for all $n$ and all $m \in [N_n]:=\{1,2,\ldots,N_n\}$. Deduce from part (vi) that $X(n):=\sum_{m=1}^{N_n}X_{n,m} \to \mathrm{Pois}(\lambda)$ in distribution as $n \to \infty$.




\item
The distribution of a random variable $X$ is called \emph{infinitely divisible} if,
for all $n \in \mathbb{N}$, there exists a sequence of independent and identically
distributed random variables $Y_1^{(n)},Y_2^{(n)},\ldots,Y_n^{(n)}$ such that
$X$ and $Y_1^{(n)} + Y_2^{(n)} + \cdots + Y_n^{(n)}$ have the same
distribution.\footnote{Examples of infinitely divisible distributions include:
	normal, Cauchy, Gamma, Poisson, geometric.}

Prove that the characteristic function $\phi$ of an infinitely divisible
distribution is nonzero for all real $t$, i.e., that $\phi(t) \neq 0$ for all $t\in \mathbb{R}$.
\\
\textbf{Hint:} Let $\phi_n$ be the characteristic function of $Y_i^{(n)}$.
Let $\psi_n(t) = |\phi_n(t)|^2$ and $\psi(t) = |\phi(t)|^2$.
\\ Step 1:  Explain why
these are also characteristic functions. 
\\ Step 2: Express $\psi_n(t)$ in terms of $\psi(t)$ and $n$ and deduce from this expression that $\lim_{n \to \infty} \psi_n(t)=g(t)$ for all $t \in \mathbb{R}$ for some $g:\mathbb{R} \to \{0,1\}$, where $g(t)$ has an expression involving $\psi(t)$ for all $t \in \mathbb{R}$. Step 3: Use that expression to show that $g$ is continuous at 0.
\\ Step 4: Apply a corollary of the continuity theorem from lecture to deduce a useful fact about $g$. 
\\
For intuition sake, observe that it is natural to expect that $Y_1^{(n)} \to 0$ in distribution as $n \to \infty$.

\item Let $X$ be a random variable.

(i) Show that  $\mathbb{E}[e^{sX}+e^{-sX}]=2 \sum_{k=0}^{\infty}\frac{s^{2k} \mathbb{E}[X^{2k}] }{(2k)!}$ for all $s \in [0,\infty)$. In particular, $\mathbb{E}[e^{sX}+e^{-sX}]<\infty$ if and only if $\sum_{k=0}^{\infty}\frac{s^{2k} \mathbb{E}[X^{2k}] }{(2k)!}<\infty$.


(ii) Recall that by Stirling's approximation  $\lim_{k \to \infty}\frac{\sqrt{2 \pi k}\left(k/e \right)^k}{k!}=1$. Use part (i) and Stirling's approximation to show that there exists $s>0$ such that $M_X(t)<\infty$ for all $t \in [-s,s]$ if and only if $\sup_{ k \in \mathbb{N} }  \frac{  \left(\mathbb{E}[X^{2k}] \right)^{\frac{1}{2k}}   }{2k}<\infty$, where $\mathbb{N}:=\{1,2,3,\ldots\}$.  
(ii) Recall that by Stirling's approximation  $\lim_{k \to \infty}\frac{\sqrt{2 \pi k}\left(k/e \right)^k}{k!}=1$. Use part (i) and Stirling's approximation to show that there exists $s>0$ such that $M_X(t)<\infty$ for all $t \in [-s,s]$ if and only if $\sup_{ k \in \mathbb{N} }  \frac{  \left(\mathbb{E}[X^{2k}] \right)^{\frac{1}{2k}}   }{2k}<\infty$, where $\mathbb{N}:=\{1,2,3,\ldots\}$. 




\item Let $d \ge 2$. For notational ease, you may assume $d=2$. The moment generating function $M_X:\mathbb{R}^d \to (0,\infty]$ of a random $d$-tuple $X=(X(1),X(2),\ldots,X(d))$ is defined to be $M_X(t)=\mathbb{E}\left[e^{t \cdot X} \right]$, where $t=(t(1),t(2),\ldots,t(d))$ and $t \cdot X=\sum_{j=1}^d t(j)X(j)$ is the standard inner product of $t$ and $X$. Recall that $X=(X(1),X(2),\ldots,X(d))\deq Y=(Y(1),Y(2),\ldots,Y(d))$ if and only if $t \cdot X \deq t \cdot Y$ for all $t \in \mathbb{R}^d$ and that $X_n=(X_n(1),X_n(2),\ldots,X_n(d)) \Rightarrow X$ if and only if $t \cdot X_n \Rightarrow t \cdot X$ for all $t \in \mathbb{R}^d$. 

Fix some norm $\| \cdot \|$ on $\mathbb{R}^d$.	Throughout all parts of the question suppose that:
\\ (*) There exists $\delta>0$ such that $M_X(t)<\infty$ for all $t \in \mathbb{R}^d$ such that $\|t\| \le \delta$.

\medskip

You are encouraged to think about how condition (*) and the statement of part (c) below simplify in the special case that $X(1),X(2),\ldots,X(d)$ are independent.

\begin{enumerate}
	\item
	Suppose that there exists $\delta'>0$ such that $M_X(t)=M_Z(t)$ for all $t \in \mathbb{R}^d$ such that $\|t\| \le \delta'$.  Show that $X \deq Z$. \textbf{Hint:} There is an easy and short solution.
	\item \textbf{Part (b) is unmarked -- Do not submit a solution to it. You may rely on it below.} Show that $\left|\mathbb{E}\left[\prod_{j=1}^{d}X(j)^{k(j)} \right] \right|<\infty$ for all $k(1),\ldots,k(d) \in \mathbb{Z}_+:=\{0,1,2,\ldots\}$. 
	\\ \textbf{Hint:} Use (*) and Part (i) of Question 6 to show that $X(j) \in L^p$ for all $p \in \mathbb{N}$, for all $j \in [d]$. Then use Holder's inequality to get that $\left|\mathbb{E}\left[\prod_{j=1}^{d}X(j)^{k(j)} \right] \right|^d \le \prod_{j=1}^{d}\mathbb{E}\left[|X(j)|^{k(j)d} \right]$. 
	
	\item Suppose that for all $k(1),\ldots,k(d) \in \mathbb{Z}_+$,  $\mathbb{E}\left[\prod_{j=1}^{d}X_n(j)^{k(j)} \right]$ is finite for all $n$ and that $\lim_{n \to \infty}\mathbb{E}\left[\prod_{j=1}^{d}X_n(j)^{k(j)} \right]=\mathbb{E}\left[\prod_{j=1}^{d}X(j)^{k(j)} \right]$. Show that\footnote{In particular, this shows that if (*) holds and  $\mathbb{E}\left[\prod_{j=1}^{d}Y(j)^{k(j)} \right]=\mathbb{E}\left[\prod_{j=1}^{d}X(j)^{k(j)} \right]$  for all $k(1),\ldots,k(d) \in \mathbb{Z}_+$, then  $X \deq Y=(Y(1),\ldots,Y(d))$.} $X_n \Rightarrow X$.
	
	\textbf{Hint:} Apply the method of moments to $t \cdot X$ for any $t \in \mathbb{R}^d$. 
\end{enumerate} 





\item
This problem concerns the method of Monte Carlo integration, which is
a method for the approximate evaluation of an integral $I=\int_0^1 f(x) \mathrm{d}x$.
%\begin{enumerate}
%\item
\\
Let $U_1, \ldots ,U_N$ be i.i.d.\ uniform
random variables on the interval $(0,1)$, and let
\[
    I_N = \frac{1}{N}[f(U_1) + \cdots + f(U_N)] \, .
\]
Suppose that $\int_0^1 f(x)^2  \mathrm{d}x < \infty$,
and let $\sigma^2 = {\rm Var}f(U_1) = \int_0^1 f(x)^2  \mathrm{d}x - I^2$.
Apply the central limit theorem to
show that $I_N$ converges to $I$ as $N \to \infty$, in the sense that for all $x \in \mathbb{R}$
\[
  \lim_{N \to \infty}  \mathbb{P} \left( |I_N - I| \leq \frac{\sigma x}{\sqrt{N}} \right)
    = \mathbb{P}(|Z| \leq x) \, ,
\]
where $Z$ is a standard normal random variable.
%\item
%Assuming that $\sigma \leq 1$,
%how large should $N$ be taken to be $95\%$ confident that $I_N$ is within
%$0.01$ of $I$?
%\end{enumerate}

\textbf{Solution.}
Consider the random variable $S_N = f(U_1) + \ldots + f(U_N)$.
Then, $m = f(U_i) = I$.
By Corollary 11.2.3 (an extension of the CLT) in the text, we have
\[
\lim_{n \to \infty} P \left( \frac{S_n - nm}{\sqrt{n v}} \leq x \right)  = \Phi(x)
\] 
if $S_n$ is the sum of random variables $X_n$ with mean $m$ and variance $v$.
In this case, we have $S_n = N I_N$, $m = I$, and $\sqrt{v} = \sigma$.
We then substitute:
\begin{align*}
    \lim_{n \to \infty} P \left( \frac{N I_N - NI}{\sqrt{N v}} \leq x \right) &= \lim_{n \to \infty} P \left( I_N - I \leq \frac{\sigma x}{\sqrt{N}} \right) = \Phi(x) = P(Z \leq x)
.\end{align*}
Therefore, the CDFs are the same, and we get:
\[
\lim_{n \to \infty}  P \left( |I_N - I| \leq \frac{\sigma x }{\sqrt{N}} \right)  = P(|Z| \leq x)
.\] 

\item
Let $X_1, X_2, \dots$ be i.i.d.\ random variables with c.d.f.\ $F$.
The \emph{empirical c.d.f.} \ $\widehat F_n : \R \to [0,1]$ is
defined by $\widehat F_n(x) = \frac 1 n \sum_{k=1}^n \1_{\{ X_k \le x\}}$.
Thus, for each $x\in\R$ and $n \ge 1$, $\widehat F_n(x)$ is a random variable.
\begin{enumerate}
\item
For each $x \in \R$, prove that $\widehat F_n(x) \to F(x)$ a.s.\ as $n \to \infty$.
\item
If $x$ satisfies $0<F(x) < 1$, prove that
\[
    \frac{\sqrt{n}(\widehat F_n(x) - F(x)) }{\sqrt{ F(x)[1-F(x)]}} \Rightarrow N(0,1) \, .
\]
\end{enumerate}

\item
Let $X_1, X_2, \dots$ be i.i.d.\ random variables with known mean $m$ and
unknown variance $\sigma^2$, both finite and with $\sigma^2 \neq 0$. Let $W_n:=\sum_{k=1}^{n} (X_k - m)^2$ and $U_n:=\begin{cases} \sqrt{W_n} & \text{if } W_n>0
	\\ 1 & \text{if } W_n=0
\end{cases}$.  Prove that
\[
    \frac{ S_n - nm }{U_n} \Rightarrow N(0,1) \, .
\]

\textbf{Solution.} 

By CLT, we have that
\[
\frac{S_n - nm}{\sqrt{n}\sigma} \implies N(0, 1)
.\] 
Now, by the strong law of large numbers, which applies since we are dealing with i.i.d. R.V.s with $(X_k - m)^2$ having finite expectation, we have
\[
\frac{W_n}{n} \to \sigma^2 \quad a.s
.\] 
Since $\sqrt{x}$ is a continuous function, we have that the limit
\[
\sqrt{W_n / n} \to \sigma \quad a.s
.\] 

Now, 
\[
U_n / \sqrt{n} = \begin{cases}
    \sqrt{W_n} / \sqrt{n} &\text{ if } W_n > 0 \\
    1 / \sqrt{n} &\text{ if } W_n = 0
\end{cases}
.\] 

Now, since $P(W_n = 0) = 0$ as $n \to \infty$, we have that $U_n / \sqrt{n} \to \sqrt{W_n} / \sqrt{n}$ a.s.
Therefore, $U_n \imlies \sigma$.

Then, we apply Slutsky's theroem to $X_n = \frac{S_n - nm}{\sqrt{n} \sigma}$ and $Y_n = U_n / \sqrt{n}$.
Since $U_n$ is never zero, $Y_n$ is never zero either.
Then,
\[
\frac{S_n - nm}{ U_n} = \frac{S_n - nm}{\sqrt{n} \sigma U_n / \sqrt{n}} = \sigma N(0, 1) / \sigma  = N(0, 1)
.\] 



\end{enumerate}


\bigskip  \noindent
{\bf Recommended problems.}
The following problems from Rosenthal are recommended but are not to be handed in:

11.5.2, 11.5.4, 11.5.6, 11.5.12

9.5.2, 11.4.2, 11.5.18
\\
For solutions to even-numbered problems see: \url{http://www.probability.ca/jeff/grprobbook.html}.


\bigskip \noindent
Additional recommended problems (\textbf{do not hand in}):
\\
%Solution in Durrett 3rd ed, 2.2.11.
\begin{enumerate}
	\item
	Prove Slutsky's Theorem: For every $n \in \mathbb{N}$ let $X_n$ and $Y_n$ be two random variables defined on the same probability space. Let $X$ be a random variable on some probability space. Let $c \in \mathbb{R}$ be a constant. Suppose that $X_n \Rightarrow X$ and that $Y_n \Rightarrow c$. Then
	\\
	(a)
	$X_n+Y_n \Rightarrow X+c$.
	(Consequently if $X_n \Rightarrow X$ and $Y_n:=Z_n-X_n \Rightarrow 0$ then
	$Z_n=X_n+Y_n \Rightarrow X$.)
	\\
	%(b)
	%Prove or disprove that, more generally, if %$X_n \Rightarrow X$ and $Y_n \Rightarrow Y$
	%then $X_n+Y_n \Rightarrow X+Y$.
	%\\
	(b) $X_nY_n \Rightarrow cX$.
	\\
	(c) Suppose $c \neq 0$ and let $z$ be an arbitrary non-zero real number. Then $X_n/Z_n \Rightarrow X/c$, where $Z_n:=\begin{cases} Y_n & \text{if } Y_n \neq 0
		\\ z & \text{if } Y_n=0
	\end{cases}$.  
	\item Show that there exists $s>0$ such that $M_X(t)<\infty$ for all $t \in [-s,s]$ if and only if there exist $C,c>0$ such that $\mathbb{P}[|X|>r] \le Ce^{-cr}$ for all $r \ge 0$.\footnote{If this conditions holds for all sufficiently large $r$ for some $C,c>0$ then it also holds for all $r \ge 0$ for some other $C,c>0$. Thus it suffices to confirm it for sufficiently large $r$.}
	
	\item
	A collection of probability measures $(\mu_j:j \in J)$ on $(\mathbb{R},\mathcal{B})$ is called uniformly integrable if for all $\varepsilon>0$ there exists $M>0$ such that $\int |x|\1_{|x|>M}\mathrm{d} \mu_j(x) < \varepsilon$ for all $j \in J$. Show that $(\mu_j:j \in J)$ is  uniformly integrable if and only if (i) $\sup_{j \in J}\int |x|\mathrm{d} \mu_j(x) < \infty$ and (ii) $(\nu_j:j \in J)$ is tight, where for $j \in J$, $\nu_j$ is the probability measure on $(\mathbb{R},\mathcal{B})$  satisfying that for every bounded Borel function $f:\mathbb{R} \to \mathbb{R}$  \[\int f(x)\mathrm{d} \nu_j(x)=\frac{\int (1+|x|)f(x)\mathrm{d} \mu_j(x)}{\int (1+|x|)\mathrm{d} \mu_j(x)} \, . \]  
\end{enumerate}	


\bigskip \noindent
Quote of the week:
\emph{
I know of scarcely anything so apt to impress the imagination as the wonderful
form of cosmic order expressed by the ``Law of Frequency of Error." The law would
have been personified by the Greeks and deified, if they had known of it. It
reigns with serenity and in complete self-effacement, amidst the wildest confusion.
The huger the mob, and the greater the apparent anarchy, the more perfect is its
sway. It is the supreme law of Unreason. Whenever a large sample of chaotic
elements are taken in hand and marshalled in the order of their magnitude,
an unsuspected and most beautiful form of regularity proves to have been
latent all along.}

\hspace*{\fill}{Francis Galton\footnote{Galton's reputation is tarnished
by his work on eugenics. \url{https://en.wikipedia.org/wiki/Francis_Galton}} describing the Central Limit Theorem [(1889) Natural Inheritance]}



\end{document}
