\documentclass[hidelinks]{article}[12pt]

\oddsidemargin 3mm
\evensidemargin 3mm
\topmargin -12mm
\textheight 630pt
\textwidth 450pt

% Some basic packages
\usepackage[utf8]{inputenc}
\usepackage[T1]{fontenc}
\usepackage{textcomp}
\usepackage{url}
\usepackage{graphicx}
\usepackage{float}
\usepackage{booktabs}
\usepackage{enumitem}
\usepackage{bbm}

\pdfminorversion=7

% Don't indent paragraphs, leave some space between them
\usepackage{parskip}

% Hide page number when page is empty
\usepackage{emptypage}
\usepackage{subcaption}
\usepackage{multicol}
\usepackage{xcolor}

% Other font I sometimes use.
% \usepackage{cmbright}

% Math stuff
\usepackage{amsmath, amsfonts, mathtools, amsthm, amssymb}
% Fancy script capitals
\usepackage{mathrsfs}
\usepackage{cancel}
% Bold math
\usepackage{bm}
% Some shortcuts
\newcommand\N{\ensuremath{\mathbb{N}}}
\newcommand\R{\ensuremath{\mathbb{R}}}
\newcommand\Z{\ensuremath{\mathbb{Z}}}
\renewcommand\O{\ensuremath{\emptyset}}
\newcommand\Q{\ensuremath{\mathbb{Q}}}
\newcommand\C{\ensuremath{\mathbb{C}}}

\DeclareMathOperator{\var}{Var}
\DeclareMathOperator{\ani}{Ann}
\DeclareMathOperator{\mnn}{Min}
\DeclareMathOperator{\spn}{span}
\DeclareMathOperator{\ch}{ch}
\DeclareMathOperator{\aut}{Aut}
\DeclareMathOperator{\ide}{Id}
\DeclareMathOperator{\iso}{Iso}
\DeclareMathOperator{\sinc}{sinc}


% Easily typeset systems of equations (French package)
\usepackage{systeme}

% Put x \to \infty below \lim
\let\svlim\lim\def\lim{\svlim\limits}

%Make implies and impliedby shorter
\let\implies\Rightarrow
\let\impliedby\Leftarrow
\let\iff\Leftrightarrow
\let\epsilon\varepsilon

% Add \contra symbol to denote contradiction
\usepackage{stmaryrd} % for \lightning
\newcommand\contra{\scalebox{1.5}{$\lightning$}}

% \let\phi\varphi

% Command for short corrections
% Usage: 1+1=\correct{3}{2}

\definecolor{correct}{HTML}{009900}
\newcommand\correct[2]{\ensuremath{\:}{\color{red}{#1}}\ensuremath{\to }{\color{correct}{#2}}\ensuremath{\:}}
\newcommand\green[1]{{\color{correct}{#1}}}

% horizontal rule
\newcommand\hr{
    \noindent\rule[0.5ex]{\linewidth}{0.5pt}
}

% hide parts
\newcommand\hide[1]{}

% si unitx
\usepackage{siunitx}
\sisetup{locale = US}

% Environments
\makeatother
% For box around Definition, Theorem, \ldots
\usepackage{mdframed}
\mdfsetup{skipabove=1em,skipbelow=0em}
\theoremstyle{definition}
\newmdtheoremenv[nobreak=true]{definition}{Definition}
\newmdtheoremenv[nobreak=true]{property}{Property}
\newmdtheoremenv[nobreak=true]{consequence}{Consequence}
\newmdtheoremenv[nobreak=true]{lemma}{Lemma}
\newmdtheoremenv[nobreak=true]{proposition}{Proposition}
\newmdtheoremenv[nobreak=true]{theorem}{Theorem}
\newmdtheoremenv[nobreak=true]{law}{Law}
\newmdtheoremenv[nobreak=true]{corollary}{Corollary}
\newmdtheoremenv{conclusion}{Conclusion}
\newmdtheoremenv{bonus}{Bonus}
\newmdtheoremenv{conjecture}{Conjecture}
\newtheorem*{recall}{Recall}
\newtheorem*{intermezzo}{Intermezzo}
\newtheorem*{notation}{Notation}
\newtheorem*{observation}{Observation}
\newtheorem*{exercise}{Exercise}
\newtheorem*{comment}{Comment}
\newtheorem*{practice}{practice}
\newtheorem*{problem}{Problem}
\newtheorem*{terminology}{Terminology}
\newtheorem*{application}{Application}
\newtheorem*{eg}{Example}
\newtheorem*{question}{Question}
\newtheorem*{previouslyseen}{As previously seen}
\newtheorem*{remark}{Remark}
\newtheorem*{observe}{Observe}
\newtheorem*{intuition}{Intuition}
\newtheorem*{solution}{Solution}

% End example and intermezzo environments with a small diamond (just like proof
% environments end with a small square)
\usepackage{etoolbox}
\AtEndEnvironment{vb}{\null\hfill$\diamond$}%
\AtEndEnvironment{intermezzo}{\null\hfill$\diamond$}%
% \AtEndEnvironment{opmerking}{\null\hfill$\diamond$}%

% Fix some spacing
% http://tex.stackexchange.com/questions/22119/how-can-i-change-the-spacing-before-theorems-with-amsthm
\makeatletter
\def\thm@space@setup{%
  \thm@preskip=\parskip \thm@postskip=0pt
}


% Exercise 
% Usage:
% \oefening{5}
% \suboefening{1}
% \suboefening{2}
% \suboefening{3}
% gives
% Oefening 5
%   Oefening 5.1
%   Oefening 5.2
%   Oefening 5.3
%\newcommand{\exc}[1]{%
    %\def\@excercise{#1}%
    %\subsection*{Excercise #1}
%}

%\newcommand{\sub-excercise}[1]{%
    %\subsubsection*{Excercise \@excercise.#1}
%}


% \lecture starts a new lecture (les in dutch)
%
% Usage:
% \lecture{1}{di 12 feb 2019 16:00}{Inleiding}
%
% This adds a section heading with the number / title of the lecture and a
% margin paragraph with the date.

% I use \dateparts here to hide the year (2019). This way, I can easily parse
% the date of each lecture unambiguously while still having a human-friendly
% short format printed to the pdf.

%\usepackage{xifthen}
%\def\testdateparts#1{\dateparts#1\relax}
%\def\dateparts#1 #2 #3 #4 #5\relax{
    %\marginpar{\small\textsf{\mbox{#1 #2 #3}}}
%}

%\def\@lecture{}%
\newcommand{\note}[3]{
    \ifthenelse{\isempty{#3}}{%
        \def\@note{#1}%
    }{%
        \def\@note{#1: #3}%
    }%
    \subsection*{\@note}
    \marginpar{\small\textsf{\mbox{#2}}}
}



% These are the fancy headers

\makeatother


% Todonotes and inline notes in fancy boxes
\usepackage{tcolorbox}

% Make boxes breakable
\tcbuselibrary{breakable}

% Verbetering is correction in Dutch
% Usage: 
% \begin{verbetering}
%     Lorem ipsum dolor sit amet, consetetur sadipscing elitr, sed diam nonumy eirmod
%     tempor invidunt ut labore et dolore magna aliquyam erat, sed diam voluptua. At
%     vero eos et accusam et justo duo dolores et ea rebum. Stet clita kasd gubergren,
%     no sea takimata sanctus est Lorem ipsum dolor sit amet.
% \end{verbetering}
\newenvironment{correction}{\begin{tcolorbox}[
    arc=0mm,
    colback=white,
    colframe=green!60!black,
    title=Remark,
    fonttitle=\sffamily,
    breakable
]}{\end{tcolorbox}}



% Figure support as explained in my blog post.
\usepackage{import}
\usepackage{xifthen}
\usepackage{pdfpages}
\usepackage{transparent}
\newcommand{\incfig}[1]{%
    \def\svgwidth{\columnwidth}
    \import{./figures/}{#1.pdf_tex}
}

% Fix some stuff
% %http://tex.stackexchange.com/questions/76273/multiple-pdfs-with-page-group-included-in-a-single-page-warning
\pdfsuppresswarningpagegroup=1

% My name
\author{Dominic Klukas}


\usepackage{hyperref}
\usepackage{url}
\usepackage{amstext}
\usepackage{amsmath,amssymb}
\usepackage{graphicx}
\usepackage{bbm}
\usepackage{enumitem}

\newcommand{\Fcal}{{\cal F}}
\newcommand{\1}{\mathbbm{1}}


\begin{document}


%\pagestyle{empty}

\noindent
{\bf Math 418/544   Assignment 7} \hfill  November 5, 2025

\bigskip \noindent
Dr.\ J.\ Hermon


Dominic Klukas, SN 64343878

\bigskip \noindent
{\bf This assignment is due in Canvas at 23:59 p.m.\ on Monday, November 17. \\
%Please read the submission instructions on Canvas. \\
\emph{Late assignments are not accepted.}}
This assignment is longer than other assignments and will have a double weight. If this is your lowest graded assignment it would still be omitted from your HW assignment average (despite having a double weight).


\begin{enumerate}


\item
A collection of random variables $(X_n:n \in \mathcal{I})$ (defined on the same probability space) is called \emph{uniformly integrable} if for all $\varepsilon$ there exists some $M=M(\varepsilon)$ such that  $\mathbb{E}[|X_n|\1_{|X_n|>M}] < \varepsilon$ for all $n \in \mathcal{I}$. Prove that $(X_n:n \in \mathcal{I})$ is uniformly  integrable if and only if $\sup_{n \in \mathcal{I}}\mathbb{E}[|X_n|]<\infty$ and for any $\varepsilon>0$, there exists a $\delta=\delta(\varepsilon)>0$
such that $E[|X_n| \1_A ] < \varepsilon$ for $n \in \mathcal{I}$, for all $A\in\Fcal$ with $\mathbb{P}(A) < \delta$.

\textbf{Solution.}

($\rightarrow$) Suppose $(X_n : n \in \mathcal{I})$ is uniformly integrable. Pick any $\epsilon > 0$, so that we choose $M$ such that $\mathbb{E} [ |X_n| \1_{|X_n| \ge M}] < \epsilon$ for all $n$. But then, for all $n$,
\[
\sup_{n \in \mathcal{I}}\mathbb{E} [ |X_n|]  = \sup_{n \in \mathcal{I}}( \mathbb{E} [ |X_n| \1_{|X_n| \ge M}]  + \mathbb{E} [ |X_n| \1_{|X_n| < M}]) \le  \epsilon + M < \infty
.\] 
We saw this technique in the proof of proposition 9.1.5 in the text.

Now, let $\epsilon > 0$. Then, there exists some $M$ such that $\forall n \in \mathcal{I}$, $\mathbb{E} [ |X_n| \1_{|X_n| \ge M}] < \epsilon / 2$ by assumption.
Let $\delta = \epsilon / 2M$.
Let $n \in \mathcal{I}$.
Let $A \in \mathcal{F}$ such that $P(A) < \delta$.
Then,
\begin{align*}
    \mathbb{E} [ |X_n| \1_A] &= \mathbb{E} [ |X_n| \1_{A \cap \{ |X_n| \ge M\}}]  + \mathbb{E} [ |X_n| \1_{A \cap \{|X_n| < M \}}] \\
    &\le \epsilon / 2 + M P(A \cap \{|X_n| <  M\} )  \\
    &\le \epsilon / 2 + M \delta \le  \epsilon / 2 + \epsilon / 2 = \epsilon
.\end{align*}
Therefore, the conclusions are satisfied: $\sup_{n \in \mathcal{I}} \mathbb{E}[|X_n|] < \infty$, and for any $\epsilon > 0$ there exists $\delta > 0$ such that $E[|X_n|\1_A] < \epsilon$ for all $n \in \mathcal{I}$, and $A \in \mathcal{F}$ with $P(A) < \delta$.

($\leftarrow$) Suppose $\sup_{n \in \mathcal{I}} \mathbb{E}[|X_n|] < \infty$, and for any $\epsilon > 0$ there exists $\delta > 0$ such that $\mathbb{E}[|X_n|\1_A] < \epsilon$ for all $n \in \mathcal{I}$, and $A \in \mathcal{F}$ with $P(A) < \delta$.
Let $\epsilon > 0$.
Let $C = \sup_{n \in \mathcal{I}} E[|X_n|]$.
Now, by assumption there exists some $\delta > 0$ such that for all $A \in \mathcal{F}$, and for all $n \in \mathcal{I}$, $\mathbb{E}[|X_n|\1_A] < \epsilon$.
Let $M = 2 C / \delta$.

Let $n \in \mathcal{I}$. Consider $\|X_n\| \1_{\|X_n\| > M}$. If $P(\|X_n\| > M) < \delta$, then $\mathbb{E}[\|X_n\| \1_{\|X_n\| > M}] < \epsilon$ and we are done. Suppose $P(\|X_n\| > M) \ge \delta$.
Then, we compute, for this fixed $n$,
\begin{align*}
    \mathbb{E}[|X_n|] &= \mathbb{E} [ |X_n| \1_{|X_n| \ge M}]  + \mathbb{E} [ |X_n| \1_{|X_n| < M}]\\
    &\ge  \mathbb{E} [ |X_n| \1_{|X_n| \ge M}] \\
    &\ge M \cdot  P(\|X_n\| > M) \ge  M \cdot \delta = 2\cdot C > \sup_{n \in \mathcal{I}} \mathbb{E} [ |X_n|]
.\end{align*}
This is a contradiction. Therefore, we must have $P(\|X_n\| > M) < \delta$, and so that $\mathbb{E}[\|X_n\| \1_{\|X_n\| > M}] < \epsilon$ and our conclusion holds.

\item

Suppose that $X_1,X_2,\ldots$ are random variables with $X_1 \ge X_2 \ge X_3 \ge \cdots \ge 0$, with $X_n(\omega) \to X(\omega)$ for all $\omega \in\Omega$, and with $\mathbb{E}X_1<\infty$.
\begin{enumerate}
\item
Prove that $\lim_{n\to\infty}\mathbb{E}X_n=\mathbb{E}X$.
\item
Give an example to show that the conclusion in part (a) need not hold if $\mathbb{E}X_1=\infty$.
\end{enumerate}


\textbf{Solution.}

\begin{enumerate}
    \item  This is a simple application of the Dominated Convergence Theorem: Since $|X_n| \le  X_1$, $\mathbb{E}[X_1] < \infty$, $X_n \to X$ w.p. 1 (indeed, pointwise!), we have that the conditions of the theorem are satisfied and $\lim_{n \to \infty} \mathbb{E}[X_n] = E[X] $.
    \item Let $X_1 = \sum_{n=1}^{\infty} 2^{n-1} \1_{[0, 2^{-n}]}$. Then, let $X_{n} = X_1 \1_{[0, 2^{-n}]}$.
        We have that for any $m \ge n$, in the intervals of type $[2^{-(m+1)}, 2^{-m}]$, we can compute
        \[
            X_n(\omega) = \sum_{s=n}^{m} 2^{s-1} = 2^{m} - 2^{n}
        .\] 
        But then, we have
        \[
        \mathbb{E} [X_n] = \sum_{m=n}^{\infty} \mathbb{E} [X\1_{\omega \in [2^{-(n+1)}, 2^{-n}]}] = \sum_{m=n}^{\infty} (2^{m}-2^{n})(2^{-m-1}) = \sum_{m=n}^{\infty} 2 - \frac{1}{2^{m-n}} = \infty
        .\] 
        However, since $\1_{[0, 2^{-n}]} \to 0$ w.p. 1, we have that $X_n \to 0$, and $E[0] = 0$, so $E[\lim_{n\to \infty} X_n] \neq \lim_{n \to \infty} E[X_n]$.
\end{enumerate}

\item
Let $X$ have a Bernoulli($\frac 12$) distribution and define the rate function
$I(z) = \sup_{s\in \R}[sz-\log M_X(s)]$.
Prove that
\[
    I(z) = \begin{cases}
    \log 2 + z\log z + (1-z)\log(1-z) & (z\in [0,1])
    \\
    \infty & \text{otherwise}.
    \end{cases}
\]
(For the case $z > \frac 12$ where we used this rate function in class for the coin flipping
example, note that the supremum over $s \in \R$ turns out to be the same as the supremum
over $s>0$ as appears in Theorem~9.3.4.)


\textbf{Solution.}

First we compute
\[
M_{X}(s) = \mathbb{E}[e^{sX}] = \frac{1}{2} ( 1 + e^{s})
.\] 

The function $f(z, s) = sz - \log M_X(s)$, so that $I(z)= \sup_{s \in \R} f(z, s)$ is in $C^{\infty}$ in $s$, so we can use the first and second derivatives to determine its optimality. Also, since continuous, the supremum is attained at some maximum.

Denote 
\begin{align*}
    \frac{\partial f}{\partial s} (s, z) &= z - \frac{1}{1 + e^{s}} e^{s}\\
    \frac{\partial^2 f}{\partial s^2} (s, z) &=  - \frac{e^{s} + e^{2s}}{(1 + e^{s})^2}
.\end{align*}
This second derivative is everywhere negative, so the function is concave in $s$.
Therefore, the maximum occurs at $\infty$ if strictly increasing, $-\infty$ if strictly decreasing, or potentially at some local maximum.

We compute the limits:
\begin{align*}
    \lim_{s \to \infty} zs - \log(1 + e^{s}) + \log 2 \approx_{s \gg 0} \lim_{s \to \infty} (z - 1)s + \log 2 \to \begin{cases}
        \infty & \text{ if } z > 1\\
        \log 2 & \text{ if } z = 1 \\
        - \infty & \text{ if } z < 1
    \end{cases}\\
    \lim_{s \to -\infty} zs - \log(1 + e^{s}) \approx_{s \ll 0} \lim_{s \to \infty}  zs \to \begin{cases}
        -\infty & \text{ if } z > 0 \\
        \log 2 & \text{ if } z = 0\\
        \infty & \text{ if } z < 0
    \end{cases}
.\end{align*}
Therefore, $\sup_{s \in \R} f(s, z) = \infty$ when $z \in (-\infty, 0) \cup (1, \infty)$.
If $z \in (0, 1)$, then both the limits to $\pm \infty$ are $-\infty$ so we look for some global maximum.
We compute:
\begin{align*}
    0 &=  f'(z, s)\\
    &= z - \frac{e^{s}}{1 + e^{s}} \\
    & \implies s = \log(z) - \log(1 - z)
.\end{align*}
If $z = 1$, we have $\sup f(1, s) = \log 2$ as $s \to \infty$.
When $z = 0$, then we have $e^{s} = 0$ which occurs as $s \to -\infty$, so we must have that $\sup f(0, s) = \log 2$ as well, as $s \to -\infty$.

But then, when $z \in (0, 1)$, we have that the maximum occurs at $s = \log(z) - \log(1 - z)$, so we plug this in to get:
\[
I(z) =  \begin{cases}
    \log 2 + z \log z + (1-z) \log (1 - z) & (z \in [0, 1])\\

    \infty & \text{ otherwise}
\end{cases}
.\] 
We observe that when $z \in \{0, 1\} $, the first equation satisfies the correct value, $\log 2$.

\item
Use Theorem~9.3.4 to obtain a numerical upper bound for the probability that the proportion
of Heads in $n$ coin flips (fair coin)
is $0.6$ or higher, for the cases $n=10,100,1000$.

\textbf{Solution.}

If the probability of heads is $0.6$ or higher, we have that $I(0.6) = \log 2 +  0.4\log(0.4) + 0.6 \log(0.6) \approx 0.02014$. Then, we plug in:
\begin{align*}
    P(S_n / n \ge  0): n = 10 &\implies  e^{- I(0.6) (10)} = e^{-6} \approx 0.818\\
    n = 100 & \implies e^{- I(0.6)(100)} \approx 0.133\\
    n = 1000 & \implies e^{-I(0.6) (1000)} \approx 1.8 \times 10^{-9}
.\end{align*}

\item
\begin{enumerate}
\item
%Durrett p.88 #2.10
If $X_n \Rightarrow X$ and $Y_n \Rightarrow c$, where $c$ is a constant (and for all $n$, $X_n$ and $Y_n$ are defined on the same probability space), then
$X_n+Y_n \Rightarrow X+c$.
(Consequently if $X_n \Rightarrow X$ and $Z_n-X_n \Rightarrow 0$ then
$Z_n \Rightarrow X$.) 
\item
Prove or disprove that, more generally, if $X_n \Rightarrow X$ and $Y_n \Rightarrow Y$
then $X_n+Y_n \Rightarrow X+Y$.
\end{enumerate}

\textbf{Solution.}

\begin{enumerate}
    \item Denote $\mu_{X} = \mathcal{L}(X)$ for a random variable $X$.
        By theorem 10.1.1, it suffices to show that $\mu_{X_n + Y_n}(-\infty, x] \to \mu_{X + c}(-\infty, x]$ for all $x$ such that $\mu_{X + c}(\{x\} ) = 0$.
        In particular, $\mu_{X + c}(\{x\} ) = 0 \iff \mu_{X}(\{x - c\} ) =0$.
        Also, we must have that $\mu_{X + c}(\{x\} ) \neq 0$ for at most countably meany $x$, since $\mu_{X + c}(-\infty, x]$ is an increasing function, and so can only have countably many discontinuities, and for each $x$ such that $\mu_{X + c}(\{x\} ) \neq 0$ , we have a discontinuity at $x$.

        All that being said, let $x$ be such that $\mu_{X + c}(\{x\} ) = 0$.
        Then, for any $\delta > 0$, there exists some $\delta > \delta' > 0$ such that $\mu_{X + c}(\{x - \delta'\} ) = \mu_{X + c}(\{x + \delta'\} ) = 0$.
        Let $\epsilon > 0$. Since $Y_n \implies c$, we have by Theroem 10.1.1 that there exists some $N$ such that for all $n > N$, both $P(|Y_n - c| > \delta) < \epsilon$.


        Now, for $F_{X_n + Y_n}(x)$, we observe that
        \begin{align*}
            \{X_n + Y_n \le  x\}  &\subseteq \{X_n \le  x - c + \delta'\}  \cup \{ |Y_n - c| > \delta'\} \\
            \{X_{n} + Y_n  \ge  x\}  & \supseteq \{X_n \le x - c - \delta' \} \cap \{|Y_n - c| \le  \delta'\} = \{X_n \le x - c - \delta'\} \setminus \{|Y_n - c| > \delta'\} 
        .\end{align*}
        Taking probabilities, this comes to:
        \begin{align*}
            F_{X_n}(x - c - \delta') - P(|Y_n - c| > \delta') &\le F_{X_n + Y_n}(x)  \le  F_{X_n}(x - c + \delta') + P(|Y_n - c| > \delta')\\
            F_{X_n}(x - c - \delta') - \epsilon &\le F_{X_n + Y_n}(x)  \le  F_{X_n}(x - c + \delta') + \epsilon
        .\end{align*}

        Then, taking $n$ to infinity, and using the fact that $X_{n} \implies X$ and $\mu_{X}(x - c - \delta') = \mu_{X}(x - c + \delta') = 0$ with 10.1.1 (3), it follows that
        \[
        F_{X}(x - c - \delta') - \epsilon \le  \lim_{n \to \infty} F_{X_n + Y_n}(x) \le  F_{X}(x - c + \delta') + \epsilon
        .\] 
        Since $\epsilon$ was arbitrary, this gives us the inequality $F_{X}(x - c - \delta') \le  F_{X + Y}(x) \le  F_X(x - c + \delta')$.
        Since $\delta$ was arbitrary, we can take $\delta \to 0$, so we have that
        \[
        \mu_{X}(-\infty, x - c) \le \lim_{n \to \infty}F_{X_n + Y_n}(x) \le  F_X(x - c)
        .\] 
        However, since $\mu_X(x - c) =  \mu_{X - c}(x) = 0$, it follows that $\mu_X((-\infty, x - c)) = F_X(x- c)$, so that the limit $\lim_{n \to \infty} F_{X_n + Y_n}(x)$ exists and equals $F_{X + Y}(x) = F_X(x - c) = F_{X - c}(x)$, completing the proof.

    \item We provide a counter example. 
        The problem relies on the fact that distributions don't rely on the underlying probability space, on which the random variables are defined.
        Let $\Omega$ be the interval $[0, 1]$, with $\mu = \mathcal{L}$ the lebesgue measure and $\mathcal{F} = \mathcal{B}$ the borel subsets of $[0, 1]$.
        Define
        \begin{align*}
            X_{n} &= \begin{cases}
                1 & \omega <  1 / 2\\
                -1 & \omega \ge 1 / 2
            \end{cases}\\
            Z = Y_n &= \begin{cases}
                -1 & \omega <  1 / 2\\
                1 & \omega \ge 1 / 2
            \end{cases}
        .\end{align*}
        for all $n$.
        Now, consider that $X_{n} \implies Z$, and $Y_n \implies Z$ since for any continuous function $f$, and for all $n$,
        \[
        \int f dX_n = \frac{1}{2} f(-1) + \frac{1}{2} f(1) = \int f dZ = \int f dY_n
        .\] 

        Then, $X_n + Y_n = 0$ for all $n$, so $X_n + Y_n \to 0$ in probability, and so also in distribution.
        However,
        \[
        Z + Z = \begin{cases}
            -2 & \omega < 1 / 2\\
            2 & \omega \ge 1 / 2
        \end{cases}
        .\] 
        Then, for any continuous function,
        \[
        \int f d(Z + Z) = 0.5 (f(-2) + f(2)) \neq f(0) = \lim_{n \to \infty} \int f d(X_n + Y_n) 
        .\] 

\end{enumerate}

\item

Let $X,X_1,X_2,\ldots$ be integer-valued discrete random variables.
Show that $X_n \Rightarrow X$ if and only if $\lim_{n\to\infty}P(X_n=m)=P(X=m)$ for
all integers $m$.


\textbf{Solution.}

We use Theorem 10.1.1, in particular condition (2): $X_n \implies X$ if for all measurable sets $A$, $\mu_{X_n}(A) \to \mu_X(A)$.

Let $A$ be a measurable set. The nature of $X_n$ (that the values it take on don't change from the integers) mean that the boundary of $A$, $\mu_{X}(\partial A)$, actually isn't information required for our proof.
Let $\epsilon > 0$.
Since $X_n$ and $X$ are integer valued random variables, we have that $\mu_X(A \cap \Z) = \mu_X(A)$ and $\mu_{X_n}(A \cap \Z) = \mu_{X_n}(A)$.
Let $M_N = \{m \in \Z : -N \le  m \le  N\}$.
By continuity of probabilities, there exists some $N$ such that $\mu_{X}(M_N) > 1 - \epsilon$, since $\mu_{X}(\Z) = 1$, since $\lim_{N \to \infty} M_N  \nearrow \Z $.
Then, since $\lim_{n \to \infty}P(X_n = m) = P(X = m)$ for all $m \in \Z$, we have that since $M_N$ contains finitely many integers, there is some $N'$ such that for all $n > N'$, $|P(X_n = m) - P(X = m)| < \epsilon / |M_N|$ for each $m \in |M_N|$.
However, this also implies that $|\mu_{X_n} (M_N) - \mu_{X}(M_N)| < \epsilon$, and so $\mu_{X_n}(M_N) > \mu_{X}(M_N) -  \epsilon > 1 - 2 \epsilon$, so $\mu_{X_n}(\Z \setminus M_N) < 2 \epsilon$.
Putting these together, we have:
\begin{align*}
    |\mu_{X_n}(A) - \mu_{X}(A)| &\le |\mu_{X_n}(A \cap M_N) - \mu_{X}(A \cap M_N)| + | \mu_{X_n}(A \cap M_N^{c})| + |\mu_{X}(A \cap M_N^{c})| \\
                                &\le \epsilon + 2\epsilon + \epsilon = 4\epsilon
\end{align*}
for all $n > N'$.
Since $\epsilon > 0$ was arbitrary, we conclude $\mu_{X_n}(A) \to \mu_{X}(A)$.
Then, applying Theorem 10.1.1 (2), we have $X_{n} \implies X$, as required.

\item
Let $\phi(t)$ be the characteristic function of the random variable $X$. Suppose
that $|\phi(t_0)| = 1$ for some $t_0 \ne 0$.
Prove there exist $a,b\in \R$ such that $P(X\in a+b\Z) = 1$.

\textbf{Solution}.
Suppose $|\phi(t_0)| = 1$ for some $t_0 \ne 0$.
Suppose that for all $a, b \in \R$, we have that $P(X \in \{a + b \Z\} ) < 1$.
Now, $\phi(t_0) = e^{i \theta}$ for some $\theta \in \R$.
Then, $P(X \in \{\theta / t_0 + 2\pi / t_0 \Z\} ) < 1$.
It follows from the definition of the exponential function that $x \in \{\theta / t_0 + 2\pi / t_0 \Z\} $ iff $e^{i t_0 x} / \phi(t_0) = 1$ iff $ \Re (e^{it_0 x} / \phi(t_0)) = 1$.
By continuity of probabilities, there must be some $\alpha > 0$ such that $P(\Re e^{i t_0 X} <1 - \alpha) > \epsilon > 0$ for some $\epsilon$.
Now, consider $\phi(t_0) = \mathbb{E} [e^{i t_0 X}]$.

We then compute (since complex integration is the sum of the real and imaginary integrals):
\begin{align*}
    \Re \mathbb{E}[e^{i t_0 X} / \phi(t_0)] &= \mathbb{E}[ \Re (e^{it_0 X} / \phi(t_0)) \1_{x \in \theta / t_0  + 2\pi / t_0 \Z}] + \mathbb{E} [\Re (e^{it_0 X} / \phi(t_0)) \1_{ 1 - \alpha \le  \Re( e^{it_0 x} / \phi(t_0)) < 1}]\\
                                        & \quad + \mathbb{E} [ \Re (e^{it_0 X} / \phi(t_0)) \1_{\Re (e^{it_0 x} / \phi(t_0)) < 1 - \alpha}]\\
    & \le  (1) P(x \in \theta / t_0 + 2\pi / t_0 \Z) + (1) P( 1-\alpha \le \Re (e^{i t_0 x} / \phi(t_0)) < 1)\\
    & \quad + (1 - \alpha) P(\Re( e^{i t_0 X} / \phi(t_0)) < 1 - \alpha)\\
    &= 1 - \alpha P(\Re e^{it_0 X} / \phi(t_0) < 1 - \alpha) < 1
.\end{align*}
However, this contradicts $\mathbb{E}[e^{i t_0 X} / \phi(t_0)] = 1$. Therefore, we must have that $P(X \in \{\theta / t_0 + 2\pi / t_0 \Z\} ) = 1$.

\item
Using characteristic functions,\footnote{You can look up the characteristic functions for
	the normal, Cauchy and exponential random variables,
	it is not necessary to perform the integrals yourself.}
prove the following:
\begin{enumerate}
	\item
	Suppose $X_i$ are independent with $N(0,\sigma_i^2)$ distributions.
	Let $S_n = X_1+\cdots+X_n$.  Then $S_n$ has distribution $N(0,\sum_{i=1}^n\sigma_i^2)$.
	In particular, if $Z_i$ has a standard normal $N(0,1)$ distribution then
	$\frac {1}{\sqrt{n}}(Z_1+\cdots +Z_n)$ also has a standard normal distribution.
	\item
	Suppose $X_i$ are independent Cauchy random variables (density $f(x) = \frac{1}{\pi}\frac{1}{1+x^2}$, $x \in \R$).
	Let $S_n = X_1+\cdots+X_n$.  Then $\frac 1n S_n$ has a Cauchy
	distribution.\footnote{Cf.\ the simulation depicted in Assignment~5.}
	
	\item
	Recall that a Geometric($p$) random variable has p.m.f. $g(k)=(1-p)^{k-1}p$ for $k \in \N$.
	Consider the following variation:  $X_n$ has p.m.f. $P(X_n=k/n) =
	(1-\lambda/n)^{k-1}(\lambda/n)$
	for $k \in \N$, with $\lambda>0$.
	Apply the Continuity Theorem\footnote{We will not finish the proof
		of the Continuity
		Theorem in class until November 17, but its statement in Theorem~11.1.14 (appearing in the lecture notes) is easy
		to understand and apply.} to prove that
	$X_n$ converges weakly to an Exp($\lambda$) random variable.\footnote{It is straightforward to prove this via the characterization of convergence in distribution in terms of convergence of the CDFs to the CDF of the limit. This is also the case for question 9.}
\end{enumerate}


\textbf{Solution.}

\begin{enumerate}
    \item We look up the characteristic function of the normal distribution: We have that $\phi_{X_i}(t) = \exp(- (1 / 2) \sigma_i^2 t^2)$.
        But then, 
        \begin{align*}
            \phi_{S_n}(t) = \phi_{\sum_{i=1}^{n} X_i}(t) = \prod_{i = 1}^{n} \phi_{X_i}(t)  = \prod_{i = 1}^{n} \exp(-\frac{1}{2} \sigma_i^2 t^2) = \exp(-\frac{1}{2} t^2 \sum_{i=1}^{n}\sigma^2) = \phi_{N(0, \sum_{i=1}^{n}\sigma^2)}(t)
        .\end{align*}
        As desired. In particular, if $Z_i$ has a standard normal distribution, then $(Z_1 + \ldots + Z_n) \sim  N(0, n)$.
        However, since the mean is zero, the variance is just $\mathbb{E}[\left( \frac{1}{\sqrt{n}} \sum_{n=1}^{n} Z_i \right)^2 ] = \frac{1}{n} \mathbb{E}[\left(\sum_{i=1}^{n} Z_i\right)^2] = \frac{1}{n} n = 1$, so that $\frac{1}{\sqrt{n}} \sum_{i=1}^{n}Z_i \sim  N(0, 1)$.
    \item We look up the characteristic function of the Cauhcy distribution, and we find that it is $\phi_{\text{Cauchy}}(t) = \exp(- |t|)$.
        But then,
        \[
        \phi_{S_n}(t) = \prod_{i = 1}^{n} \phi_{X_i}(t) = \exp(-n|t|)
        .\] 
        Next, we compute:
        \[
        \mathbb{E}[\exp(i t \frac{1}{n} S_n)] = \mathbb{E} [\exp(i (t / n) S_n)] = \exp(-n | (t / n)|)  = \exp(- |t|) = \phi_{\text{Cauchy}}(t)
        .\] 
        Here, we used the fact that $n \in \N$, so $n \ge  1$ and $|t / n| = |t| / n$.
        Therefore, $\frac{1}{n} S_n \sim  \phi_{\text{Cauchy}}(t)$.

    \item We look up the characteristic function of $g(k)$, and find that it is $\frac{p}{e^{-it} - (1- p)}$.
        Now, if for $X_n$, we have $P(X_n = k / n) = (1 - \lambda / n)^{k - 1} (\lambda / n)$.
        But then, observe that $P(n X_n = k) = (1 - \lambda / n)^{k-1} (\lambda / n)$, so $nX_n$ is a geometric random varaible for $p = \lambda / n$. (Naturally, this requires that $n > \lambda$, but this should not be a problem since we are fixing $\lambda$ and considering the limit of $n$).
        Then, we have
        \begin{align*}
            \phi_{X_n} (t) = \mathbb{E}[\exp(i t X_n)] = \mathbb{E} [\exp(i (t / n) (n X_n))] &= \frac{\lambda / n}{ e^{-i(t / n)} - ( 1 - \lambda /n )}\\
            &= \frac{\lambda}{n (e^{-i(t / n)} - 1) + \lambda }
        .\end{align*}

        However, then, $\lim_{n \to \infty} \phi_{X_n}(t) = \frac{\lambda}{ \lim_{n \to \infty} n(e^{-i (t / n)} - 1) + \lambda} $.
        However, observing that $n(e^{-i ( t / n)} - 1) = \frac{e^{-it (0 + 1 / n)} - e^{-i t (0)}}{1 / n}$, we can see that this is just the limit of a difference quotient of the function $e^{-itx}$ evaoluated at $x = 0$, along the series of poitns $\{1 / n\}_{n \in \N}$.
        Since $e^{-itx}$ is differentiable at $x = 0$, every such limit evaluates to the derivative, so we conclude that $\lim_{n \to \infty} n(e^{-i (t / n)} - 1) = -it$.
        Therefore, $\lim_{n \to \infty} \phi_{X_n}(t) = \frac{\lambda}{\lambda -it}$.
        We look up the characteristic function of the exponential distribution, and see that it is $\frac{\lambda}{\lambda - it}$.
        We apply the Continuity Theorem, to see that since we have $\phi_{X_n} \to \phi_{\text{Exp}(\lambda)}$ then $X_{n} \implies \text{Exp}(\lambda)$, as desired.
\end{enumerate}

\item
Let $Y_1,Y_2,\ldots$ be i.i.d.\ random variables
(on the same probability space),
each of which takes any value in $\{0,1,2,3,4,5,6,7,8,9\}$ with equal probability
$\frac{1}{10}$.  Let
\[
X_n = \sum_{i=1}^n \frac{Y_i}{10^i}.
\]
Apply the Continuity Theorem to
prove that $X_n$ converges in distribution to a uniform random variable on $[0,1]$.

\textbf{Solution.}

We compute, using $j$ instead of $i$ to avoid confusion with $i = \sqrt{-1}$:
\[
\phi_{\frac{Y_j}{10^{j}}}(t) = \mathbb{E} \left[ \exp( i t Y_j / 10^{j}) \right] = \sum_{m = 0}^{9} \exp(i t m / 10^{j}) = \frac{1}{10} \frac{1 - e^{it / 10^{j-1}}}{1 - e^{i t / 10^{j}}}
.\] 

Then,
\begin{align*}
    \phi_{X_n}(t) = \prod_{j = 1}^{n} \phi_{\frac{Y_j}{10^{j}}}(t) = \frac{1}{10^{n}} \prod_{j = 1}^{n} \left( \frac{1 - e^{it / 10^{m-1}}}{1 - e^{it / 10^{m}}} \right)  &= \frac{1}{10^{n}} (1 - e^{it}) \prod_{j = 1}^{n - 1}\left( \frac{1 - e^{it / 10^{j}}}{1 - e^{it / 10^{j}}} \right) (1 / (1 - e^{it / 10^{n}}))\\
                                                                                                                                                                          &= \frac{1}{10^{n}} \frac{1 - e^{it}}{1 - e^{it / 10^{n}}}
.\end{align*}
Similar to last time, we sniff out a difference quotient to compute the limit:
\[
\lim_{n \to \infty} \frac{1}{10^{n}} \frac{1 - e^{it}}{1 - e^{it / 10^{n}}} = \frac{1 - e^{it}}{\lim_{n \to \infty} -\frac{ e^{it (0 + 1 / 10^{n})} - e^{it (0)}}{1 / 10^{n}} } = \frac{1- e^{it}}{- it} = \frac{i}{t} (1 - e^{it})
.\] 

We compute the characteristic formula of the uniform random variable on $[0, 1]$:
\[
    \mathbb{E}[\exp(it U_{[0, 1]})] = \int_{0}^{1} e^{it x} dx = \frac{1}{it} e^{it x} \bigg |_{x = 0}^{1} = \frac{i}{t} ( 1 - e^{it})
.\] 
Therefore, $\lim_{n \to \infty} \phi_{X_n} = \phi_{U_{[0, 1]}} $, so by the Continuity Theorem, we have that $X_n \implies U_{[0, 1]}$, as desired.

\end{enumerate}


\bigskip  \noindent
{\bf Recommended problems.}
The following problems from Rosenthal are recommended but are not to be handed in:

9.5.4, 9.5.10, 9.5.15, 9.5.16, 10.3.1, 10.3.2, 10.3.3, 10.3.4, 10.3.6. 
10.3.8, 10.3.9, 10.3.10.
\\
For solutions to even-numbered problems see: \url{http://www.probability.ca/jeff/grprobbook.html}.

\bigskip\noindent
More recommended problems (not to be handed in):
\begin{enumerate}[label=\Alph*)]
\item
Let $X$ be a Poisson random variable with parameter $\lambda >0$.  Compute the moment
generating function of $X$ and differentiate it to compute the mean and variance of $X$.

\item
Let $\alpha,\lambda>0$.  A random variable $X$ has a Gamma($\alpha,\lambda$)
distribution if its density function is given by
\[
    f(x) =
    \begin{cases}
    \frac{1}{\Gamma(\alpha)}\lambda^\alpha x^{\alpha -1} e^{-\lambda x} & ( x\ge 0)
    \\
    0 & (x<0).
    \end{cases}
\]
Here $\Gamma(\alpha) = \int_0^\infty x^{\alpha-1}e^{-x}dx$.  Recall that $\Gamma(n)=(n-1)!$
for $n\in\N$.
Note that
an Exp($\lambda$) random variable has the same distribution as a Gamma($1,\lambda)$ random
variable.
\begin{enumerate}
\item
Prove that the sum of $n$ i.i.d.\ Exp($\lambda$) random variables has a Gamma($n,\lambda$)
distribution in two ways:  by induction on $n$ using a calculation of the density function using the convolution formula and by comparison of
moment generating functions or of characteristic functions.\footnote{We have not proved it, but it is the case that
if two random variables have the same moment generating function, which is defined on
a neighbourhood of the origin, then the two random variables have the same distribution
(see Theorem~11.4.3).
A related statement that we will see sooner in class is Corollary~11.1.7.}
\item
Prove in two ways (as above) that the sum of independent Gamma($\alpha,\lambda$)
and Gamma($\beta,\lambda$) random variables is Gamma($\alpha+\beta,\lambda$).
\\
A useful fact:
$\int_0^x t^{r-1}(x-t)^{s-1} dt = x^{r+s-1}\frac{\Gamma(r)\Gamma(s)}{\Gamma(r+s)}$ for $r,s,x>0$.
\end{enumerate}
\end{enumerate}


\bigskip \noindent
Quote of the week:
{\emph{``The other girls study mathematics," Ai-ming said, trying again.}

\emph{``That's what we need!" the vendor said, smacking her chopsticks
against the metal pot.  ``Real numbers.  Without real numbers, how
can we fix our economy, make plans, understand what we need?  Young
lady, I don't mean to be rude but you should really think about
studying mathematics, too."}

\emph{``I will."}}


\hspace*{\fill}{Madeleine Thien in \emph{Do not say we have nothing}}

\bigskip \noindent
Second quote of the week:
\emph{Mathematicians are like Frenchmen:  whatever you
	say to them they translate into their own language and forthwith it
	is something entirely different.}
\hspace*{\fill}{J.W.\ Goethe}

\end{document}
