\note{1}{Thu 04 Jan}{First Summary}

\section{First Notes}

\subsection{The Activity and Art Of Reading}

\begin{itemize}
	\item Reading is still relevant. Interesting picture about how TV can make your opinions mindless
	\item Active reading: Key pivoting point is that an active reader demands more from the text.
	\item Writing and reading properly is like a game of catch, where the communication is being sent/caught.
	\item What does "catching the communication involve"? If you do not initially understand what the author is telling you, you will have to put in serious effort to elevate your level understanding to the of the author's. The text illustrates this by comparing reading for information and reading for understanding.
	\item (Side note) when the author understands something, I imagine he has a picture in his head of what his ideas are. Communication is when you mould your own mind to have the same shape as the authors.
	\item Enlightenment is exactly that: understanding what the ideas are, how they are connected, and why they matter, and seeing and using them in the same way the author does.
	\begin{itemize}
		\item You can understand this more or less deeply: think abstract algebra. Understanding a definition is not enough: knowing all the cases where it shows up and why the definition was motivated in the first place along with a bunch of pictures of what they look like in difference cases: that is understanding. 
	\end{itemize}	
	\item Learning is gaining enlightenment. Now, how do we learn? Our mind operates on - by the process of thinking - certain materials, such as the environment, our observations, our thoughts, our experiences, or reading/lecture materials, and by thinking on these materials, we form ideas and gain enlightenment.
\end{itemize}

\subsection{The Levels of Reading}

\begin{itemize}
	\item There are 4 different levels of reading which are structured as nesting dolls. 
	\item Elementary reading involves translating ink and paper into language in our heads; this level of reading answers, what does the sentence say? 
	\item Inspectional reading answers the question, what is the book about? This involves quickly reading certain parts of the book to gain as much info quickly as possible.
	\item Analytical reading is at the center of the art of reading: coming to understand the author's message.
	\item Synoptic reading is when you read multiple books on a subject matter and develop a well rounded understanding of it.
\end{itemize}

\subsection{The First Level of Reading: Elementary Reading}
\begin{itemize}
	\item There is new-found interest in reading education, for three reasons:
		\begin{itemize}
			\item  The American dream prescribes that everyone, no matter what their background or level of motivation, learns to read.
			\item New teaching methods have recently been developed and tested (in the 1970's)
			\item Americans have a habit of blaming the education system, so research into how to improve the education system, especially reading education, has been given specific attention.
			\item More people are going to college than ever before, so teachers are forced to continue thinking about how to teach people to read.
		\end{itemize}
	\item What are the stages in learning how to read?
		\begin{itemize}
			\item  You must first be emotionally, socially, physically, and mentally ready to do so.
			\item You learn to associate symbols with sounds and language, and you rapidly expand your vocabulary to read simple texts
			\item You begin to read more widely, and learn to read in a certain field
			\item You enhance your ability to read, by learning to read more widely and extract ideas from the texts which can be compared to other texts.
		\end{itemize}
	\item These stages are all part of the first level of reading. Sometimes people barely get through the first level of reading by the time they graduate college. We can do better than this! Also this book does not present some nifty new method, but it is common sense.
\end{itemize}
\subsection{The Second Level: Inspectional Reading}
\begin{itemize}
	\item  First, judge the book by its cover (haha) table of contents, index (look at the pages which contain important words), publisher's blurb, and preface. Try to answer what the book is about, why it is special, what the author cares about (why he wrote it) and what niche the book fills.
	\item Next, we skim the book systematically, to further investigate the  main message of the book, and the book's structure.
	\item The second type of inspectional reading is when you read through the entire book without stopping and muddling over parts you don't understand for too long, simply soaking in the parts you do understand.
	\item On speed reading courses: Good, but they simply improve your mechanics. People "sub-vocalize", and fix/regress their eyes too much. You can correct these habits with training; your brain is capable of more, if you train your eyes to give your brain the info it needs. However, for many texts you actually have to stop and think about the stuff you just saw, and speed reading won't help you there. You should read a text as quickly or as slowly as appropriate. The aim of analytical reading is to comprehend the book thoroughly. 
\end{itemize}
\subsection{How to Be a Demanding Reader}
\begin{itemize}
	\item We are active, demanding readers when we ask questions of the text. Also, it makes a huge difference to them whether or not they read the book they are reading.
	\item The main questions we ask are as follows:
		\begin{itemize}
			\item What is the book about?
			\item What is being said in detail, and how?
			\item Is the book true?
			\item What of it?
		\end{itemize}
	\item Marking the book up helps you ask those questions and to think about the text. You can underline/
        asterisk parts of the book to outline the author's main points.
	\item There are different types of notes you can take about a book. You can take notes about it's structure, conceptual notes about what it says, and finally notes comparing multiple different books (this seems more like writing an essay to me). I think the first is like creating a map, which will show you where you need to go to get certain information from the text.
	\item The activities in reading well are a habit like any other. It's important to know this, and to spend active effort trying to practice the rules and skills involved in good reading, even if you do them separately.
\end{itemize}
\subsection{Pigeonholing a book}

\begin{itemize}
	\item Knowing what to ask of a book is a good idea, and knowing what type of book it is helps you to know the authors intentions, and so what you should ask of a book. Having predetermined categories in your mind about what a book is about also helps you to know something about the book simply by knowing what its category is.
	It also helps you ask the right questions of a book.
	\item First, we distinguish between fiction and non-fiction.
	\item We discover the classification of a book through inspectional reading. This chapter gives examples where we can learn what the book is about through carefully reading the title.
	\item We look at a distinction between practical and theoretical books. Practical books teach you how to act, and theoretical books delight simply in pure knowledge, having nothing to do with pure action.
	This is useful to distinguish because it allows us to know how we are to react: are we supposed to try to implement what we are being told, or are we simply supposed to try to understand something for our own interest?	\item We then break down the subject of theoretical books into historical, philosophical, and scientific books.
	\item Theoretical books are interested in knowing something. Practical books are interested in applying that knowledge to solve some problem.
	\item Historical books deal with the past and tell a story. Philosophical books deal with things that be experienced by everybody. Scientific books require special knowledge and experiences.
	\item Different types of books can be seen as different in the sense the lectures of different subjects are different. A history lecture will be taught as a story. A philosophy lecture different from a math lecture. In the same way, the way we are taught and the way we learn each of these subjects is different, which is why we should classify the book type, so we can determine this.
	\item How is this useful to me? The Bible has different genres. This is a specific example of this. I will do this subconsciously without thinking about it, but doing it intentionally won't hurt: reading different types of literature in the Bible differently. When studying history, we learn what is important.
	\item Knowing whether a book in the bible is practical or theoretical. Also, knowing whether a book is historical. A historical book concerns itself with events in the past, and explains in particular the significance of the events in the past. None of the events would be recorded if the author didn't think they had an importance of some sort, in explaining how they came to be here.
\end{itemize}

\subsection{X-raying a book} 
\begin{itemize}
    \item Every piece of art has a unity; a reason for all of its parts being tied together.
        In the case of a good book, this is the main point or theme of the book.
    \item In order to prove that you know what this unity is, you have to show how the parts of the book
        come together to build up this unity. Our two rules of reading are this: state the unity, and the parts
        and then show how they fit together.
    \item We look at some examples. Sometimes author tells you: but you still have to think and make your own.
    \item For finding the multiplicity of a book: outline the book like a tree in graph theory.
        Remark in the level of detail to go into, (commentaries being an extreme example). Your outline
        might be different from the author's outline.
    \item Writing should involve a unity and a good outline. But in elevating the reader to your level of understanding
        you also need flesh on your bones.
    \item You should also write questions which the text answers; or find out what questions perhaps the author
        was trying to answer as he wrote the book.
    \item All of these stages answer the question, what is the book about as a whole?
\end{itemize}

\subsection{Coming to Terms With an Author}
\begin{itemize}
    \item You have to agree with an author (come to terms - agreement) about the definitions of his words
    \item Two parts: find which words are important. Which ones do you not understand: likely that the author is using them in an unusual way, and you will have to find out how he is using them unusually: find out what his term is.
    \item For the second stage of analytical reading, we are trying to discover exactly what the author is saying. There are two phases to this: we have to find his thoughts in the language, since language is not a perfect medium of thought, and then we can work on thinking about what his thoughts mean.
    \item The first step in this process is dealing with important words. First, find them. Next, figure out what term (the idea, the abstract thought being expressed by the word) means.
    \item One term can be expressed by many words.
    \item One word can be used to express many words.
    \item We find the meanings of the terms through context. Phrases can also be used to express terms.
    \item Books are like puzzles: you have to know what the terms mean. Think of the bible: faith, kingdom of heaven, etc.
\end{itemize}

\subsection{Determining an Author's Message}
\begin{itemize}
    \item We go back to the business deal analogy: we came to terms. But before coming to terms, you have to make propositions: propositions of intent. When it comes to writing, we do the opposite: we first agree on terms to use, and then the author states his propositions of knowledge (what he claims is true that he knows). The author may also state his intent (at the beginning of the book, in the preface for instance, where says what he is aiming to do with this book).
    \item Propositions are related to sentences in the same way terms are related to words.
    \item Arguments are the next logical unit above propositions which are above terms. Arguments revolve around key propositions that the author makes, and aim to prove the key propositions.
    \item When outlining a book, the arguments are at the bottom of the tree. Here, we are ``working up'' to our outline.
    \item A sentence may contain multiple propositions, due to ambiguity or due to the sentence being super long.
    \item You can find key sentences (the ones which will contain the important propositions) in a few ways
        \begin{itemize}
            \item  They contain key terms
            \item They puzzle you
            \item The author underlines them
            \item They are a conclusion or premise in an argument
            \item Just because it puzzles you doesn't mean it is important! Make sure you follow the author's ideas.
        \end{itemize}
    \item You can come to understand key sentences - and through it, know the propositions, from context, by analyzing the grammar, and by thinking hard about it by trying to state it in your own words, muddling over terms, and thinking up examples.
    \item You can find the arguments by linking together they key propositions, and possibly looking at the structure of the text itself. If you see reasons and evidence look for a conclusion. If you see a conclusion look for reasons and evidence. Think of deductive/inductive reasoning, and look for assumptions/things being proved/things taken as self evident. Your goal is to find the thing being proved (that is important to the author) and how he proves it so you can decide whether you agree with him or not.
    \item The ``solutions'' are the solutions to the problems the author was trying to solve with his book. He should have solved him with his arguments.
\end{itemize}

\subsection{Criticizing a Book Fairly}
\begin{itemize}
    \item You have the opportunity to talk back to a book: no on is stopping you from talking to yourself.
    \item You have the obligation to talk back to a book: The knowledge should interact with what you already know to be true, so you can think hard yourself if what the author said is true, and then give him a response. You have (after understanding the book) lifted yourself up to the status of the author's peer. Being teachable means that the stuff your are being taught actually makes sense to you, and so you have to criticize it.
    \item Rhetoric is the art of stringing logic together to convince/teach someone else effectively. You have to also know whether or not the writer is an empty orator or actually using his rhetorical true to further the spread of truth.
    \item Make sure you understand a book first. Next, do not disagree for the sake of disagreeing. Finally, rational men can agree!
\end{itemize}

\section{One Year Later}

\subesection{Preface}
\begin{itemize}
    \item Highlights the reason the book was written, rewritten, and foreshadows what the author will say in the book.
    \item When reading, look out for:
        \begin{itemize}
            \item References to speed reading
            \item References to the way reading is taught
            \item Other cultural references: More people going to college, television and radio.
            \item In particular, how the art of reading applies to these societal trends.
            \item A better understanding of what reading is, and it's parts.
            \item Synoptic reading, in response to parodies.
        \end{itemize}
\end{itemize}

\subsection{The Activity and Art of Reading}

\begin{itemize}
    \item Introduction: We define a reader. A reader is someone who gains understanding of the world through books. TV does not make reading obsolete, because it does not increase our understanding, but gives us prepackaged opinions.
    \item We define reading, and explore the rationale behind the book. Reading is when the author communicates with the reader through the book. The goal is that the reader receives the author's communicated intention. The reader does so more effectively by reading more actively and performing each of the acts more skillfully. I presume that this book will teach us all of these actions.
    \item Reading: receiving the Author's intended message. However, there are two situations. Either you understand the author quickly and easily, or you are puzzled when you first read the book. If you are puzzled, you have to think when you read to get yourself to the point where the Author's intended message finally takes it's shape in your head. This is skilled reading.
    \item This puzzlement happens when the author's understanding is greater than your own, and through practicing "the art of reading" you approach the Author's understanding. The line between reading things that are at once intelligible to you (for information) and reading for understanding is hazy.
    \item Enlightenment is understanding why the author says something and what he means by it, which is what it means to come to understanding. Learning by discovery, learning by being taught, and learning by reading are all similar processes, where the mind operates on material to come to understanding by applying thought (analysis and reflection), exercising memory, observation, and imagination, which is a very active activity in every case. Reading is learning from an absent teacher (the author), which is difficult because when you are puzzled you have to ask and answer your questions yourself.
\end{itemize}
